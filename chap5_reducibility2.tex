\input{header}

\AtBeginSubsection[]
{
	\begin{frame}<beamer>
		\frametitle{Outline}
		\tableofcontents[current,currentsubsection]
	\end{frame}
}

\begin{document}

\begin{frame}[allowframebreaks] \frametitle{$REGULAR_{\text{TM}}$ Undecidable}
\begin{itemize}
\item For a given TM, we hope to check if there exists
  an equivalent finite automaton
\item The problem can be formulated as follows
  \begin{equation*}
    \begin{split}
      REGULAR_{\text{TM}} 
      = \{
      \langle  M\rangle \mid { } & M \mbox{ is a TM and} \\
      & L(M)
      \text{ is a regular language}    \}
    \end{split}
  \end{equation*}
\item As before, assume this language is decidable
  and has a decider $R$
\item We construct a decider $S$ for $A_{\text{TM}}$
  as follows
  \begin{enumerate}
  \item Design a TM $M_2$ such that it recognizes
    \begin{equation}
      \begin{cases}
        \text{a regular language}  & \text{if $M$ accepts $w$}  \\
        \text{a non-regular language} & \text{if $M$ rejects $w$}
      \end{cases}
    \label{eq:M2}
    \end{equation}
\item Run $R$ on input $\langle M_2 \rangle$
  \item If $R$ accepts, {\em accept}; if $R$ rejects,
    {\em reject}
  \end{enumerate}
\item Then we have
  \begin{equation*}
    \begin{cases}
     \text{$S$ accepts} & \text{if $M$ accepts $w$} \\
     \text{$S$ rejects} & \text{if $M$ rejects $w$}
    \end{cases}
  \end{equation*}
  \item Thus $S$ is a decider for $A_{\text{TM}}$, a
  contradiction
\item Now we give a specific design of $M_2$ so that
\eqref{eq:M2} holds
\item We let $M_2$ recognize
  \begin{equation*}
    \begin{cases}
      \Sigma^* & \text{if $M$ accepts $w$} \\
      0^n1^n, \forall n \geq 0 & \text{if $M$ rejects $w$}
    \end{cases}
  \end{equation*}
    \item Note that $\Sigma^*$ is a regular
    language, but $0^n1^n, \forall n\geq 0$ is not
  \item The implementation is as follows
  \item [] On input $x$:
    \begin{enumerate}
    \item If $x$ has the form $0^n1^n$, {\em accept}
    \item If $x$ does not have the form $0^n1^n$, run
      $M$ on input $w$ and {\em accept} if $M$ accepts $w$
    \end{enumerate}
  \item We see that if $M$ accepts $w$, then any
    $x \in \Sigma^*$ is accepted
  \item On the other hand, if $M$ does not accept $w$,
    only $0^n1^n, \forall n \geq 0$ are accepted
\end{itemize}\end{frame} 

\begin{frame}[allowframebreaks] \frametitle{$EQ_{\text{TM}}$ Undecidable}
\begin{itemize}
\item So far, our strategy for proving languages undecidable
  involves a reduction from $A_{\text{TM}}$
\item Sometimes reducing from some other undecidable language
  is more convenient
\item Here we show an example by considering
  \begin{equation*}
    \begin{split}
      EQ_{\text{TM}} = \{
      \langle  M_1, M_2 \rangle \mid { }  & M_1 \text{ and } M_2 \text{ are TMs
        and } \\
      & L(M_1) = L(M_2)\}
    \end{split}
  \end{equation*}
\item Assume $EQ_{\text{TM}}$ is decidable and has a decider
  $R$
\item We construct a decider $S$ for $E_{\text{TM}}$ as follows
\item For input $\langle M \rangle$, where $M$ is a TM:
  \begin{enumerate}
  \item Run $R$ on input $\langle M, M_1 \rangle$,
    where $M_1$ is a TM such that
    \begin{equation*}
    L(M_1) = \emptyset
  \end{equation*}
\item If $R$ accepts, {\em accept}; if $R$ rejects, {\em reject}
  \end{enumerate}
\item For $M_1$, we simply let it reject any input string.
  Recall we learned how to design an NFA for $\emptyset$
\item But $E_{\text{TM}}$ is undecidable by an earlier proof.
  Thus $EQ_{\text{TM}}$ is undecidable
\end{itemize}
\end{frame}

\end{document}

%%% Local Variables:
%%% mode: latex
%%% TeX-master: t
%%% End:
