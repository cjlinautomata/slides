\input{header}

\AtBeginSubsection[]
{
	\begin{frame}<beamer>
		\frametitle{Outline}
		\tableofcontents[current,currentsubsection]
	\end{frame}
}

\begin{document}

\begin{frame}[allowframebreaks] \frametitle{Variants of TM}
  \begin{itemize}
\item We will discuss some variants that have the same power

\item
  The robustness of a type of machines means that its reasonable
variants have the same 
power

\item
  [] Not a strict definition though

\item Example
  \begin{equation*}
\delta: Q \times \Gamma\rightarrow Q \times \Gamma 
\times \{L, R, S\}
\end{equation*}
\item
  [] $S$: stay at the same position

\item
  It's equivalent to TM because $S$ can
  be implemented by $L$ \& $R$ moves:
  \begin{equation*}
      q_1,a \rightarrow q_2,b, S
    \end{equation*}
    can be replaced by several rules
    \begin{align*}
q_1,a &\rightarrow q_3, b,R \\
q_3, ? &\rightarrow q_2, ?, L, \forall ? \in \Gamma 
\end{align*}
\end{itemize}
\end{frame}

\begin{frame}[allowframebreaks] \frametitle{Multi-tape TM}
  \begin{itemize}
\item several tapes
\item input: put into tape 1

\item
  [] others: blank
\item transition is applied on all tapes simultaneously
  \begin{eqnarray*}
&& \delta: Q \times \Gamma^k \rightarrow 
Q \times \Gamma^k \times \{L, S, R\}^k     \\
&& \delta(q_i, a_1, \ldots, a_k)
= (q_j, b_1, \ldots, b_k, L, R, \ldots, L),
  \end{eqnarray*}
where $k$ is the number of tapes
\item Looks more powerful but equivalent to single-tape TM (discussed later)
\item Note that we allow $S$ here though it can be removed


\end{itemize}\end{frame} \begin{frame}[allowframebreaks] \frametitle{Example}

  \begin{itemize}
\item Job: given $w = 0^{2n}, n \geq 1$
$\Rightarrow$ generate $ww$ in the end

\item
  [] Note that we also need to check if $|w|$ is even

\item Idea
  \begin{itemize}
  \item copy $w$ to the second tape
  \item check if $|w|$ is even
  \item copy $w$ from the second tape to the first
  \end{itemize}

  
\item State diagram

  \end{itemize}

\begin{tikzpicture}
  \node[state,initial below] (q_0) {$q_0$};
  \node[state,right of=q_0,xshift=-0.3cm] (q_1) {$q_1$};
  \node[state,right of=q_1,xshift=-0.3cm] (q_2) {$q_2$};
  \node[state,right of=q_2,xshift=-0.5cm] (q_3) {$q_3$};
  \node[state,right of=q_3,xshift=-0.3cm] (q_4) {$q_4$};
  \node[state,below of=q_4,yshift=0.5cm] (q_a) {$q_a$};
  \path (q_0) edge[above] node{
    \begin{tabular}{l}
      $0 \rightarrow$ R\\
      $\sqcup \rightarrow$ R      
    \end{tabular}
  } (q_1);
  \path (q_1) edge[loop below] node{
    \begin{tabular}{l}
      $0 \rightarrow$ R\\
      $\sqcup \rightarrow 0,$ R      
    \end{tabular}
  } (q_1);
  \path (q_1) edge[above] node{
    \begin{tabular}{l}
      $\sqcup \rightarrow$ L\\
      $\sqcup \rightarrow$ L      
    \end{tabular}
    } (q_2);
  \path (q_2) edge[above, bend left] node{
    \begin{tabular}{l}
      $0 \rightarrow$ R\\
      $0 \rightarrow$ L
    \end{tabular}
    } (q_3);
  \path (q_3) edge[below, bend left] node{
    \begin{tabular}{l}
      $\sqcup \rightarrow $ L\\
      $0 \rightarrow$ L      
    \end{tabular}
    } (q_2);
  \path (q_3) edge[above] node{
    \begin{tabular}{l}
      $\sqcup \rightarrow$ 0, R\\
      $\sqcup \rightarrow$ 0, R      
    \end{tabular}
    } (q_4);
  \path (q_4) edge[loop above] node{
    \begin{tabular}{l}
      $\sqcup \rightarrow$ 0, R\\
      $0 \rightarrow$ R      
    \end{tabular}
    } (q_4);
  \path (q_4) edge[left] node{
    \begin{tabular}{l}
      $\sqcup \rightarrow$ R\\
      $\sqcup \rightarrow$ R
    \end{tabular}
    } (q_a);
\end{tikzpicture}

\begin{itemize}
\item $q_0$ to $q_1$:
  
\item [] let $\sqcup$ be used to indicate the beginning of the second
  tape
\item loop at $q_1$:
\item [] copy $w$ to the second tape
\item $q_2 \rightarrow q_3 \rightarrow q_2$: 
  \begin{enumerate}
  \item check if length is $2n$    
  \item head of the first tape zig-zag between last 0 and the
    $\sqcup$ after
  \item head of the second tape moved to the beginning
\end{enumerate}

\item If length $2n$, we should be at $q_3$
instead of $q_2$ when reaching the beginning of the
second tape
\item $q_4$: copy $w$ from the second tape to the first
\item Note that we have a deterministic machine. At $q_0$ we
  should have
  \begin{equation*}
    \begin{split}
      & \delta(q_0, 0, 0), \delta(q_0, 0, \sqcup)\\
      & \delta(q_0, \sqcup, 0), \delta(q_0, \sqcup, \sqcup)\\      
    \end{split}
  \end{equation*}
Those not specified go to $q_r$
\item Example: input 0000

\renewcommand{\arraystretch}{1.7}  
  \begin{tabular}{llll}
$ q_0
\begin{smallmatrix}
  0 &0 & 0 & 0 \\
  \sqcup & \sqcup & \sqcup & \sqcup
\end{smallmatrix}
$
&
$
\begin{smallmatrix}
  0 \\
  \sqcup 
\end{smallmatrix}
q_1
\begin{smallmatrix}
  0 & 0 & 0 \\
  \sqcup & \sqcup & \sqcup
\end{smallmatrix}
$
&           
$\cdots
$
&
$
\begin{smallmatrix}
  0 & 0 & 0 & 0\\
  \sqcup & 0  & 0 & 0
\end{smallmatrix}
q_1
\begin{smallmatrix}
  \sqcup\\
  \sqcup 
\end{smallmatrix}
$
\\
$
\begin{smallmatrix}
  0 & 0 & 0 \\
  \sqcup & 0  & 0 
\end{smallmatrix}
q_2
\begin{smallmatrix}
  0\\
  0
\end{smallmatrix}
$
&
$
\begin{smallmatrix}
  0 & 0 & & 0 & 0 & q_3\\
  \sqcup & 0 & q_3 & 0 & 0 &
\end{smallmatrix}
$
& 
$
\begin{smallmatrix}
  0 & & 0 & 0 & q_2 & 0\\
  \sqcup & q_2 & 0 & 0 &&  0 
\end{smallmatrix}
$
&
$
\begin{smallmatrix}
  & 0 & 0 & 0 & 0 & q_3\\
  q_3 & \sqcup & 0 & 0 & 0 
\end{smallmatrix}
$
\\
$
\begin{smallmatrix}
   0 & & 0 & 0 & 0 & 0 & q_4\\
  0 & q_4 & 0 & 0 & 0 & \sqcup
\end{smallmatrix}
$
& 
$\cdots$
&
$
\begin{smallmatrix}
   0 & 0 & 0 & 0 & 0 & 0 & 0 & 0 & q_4\\
  0 &  0 & 0 & 0 & q_4 &&&&
\end{smallmatrix}
$
&
% $
% \begin{smallmatrix}
%    0 & 0 & 0 & 0 & 0 & 0 & 0 & 0 & \sqcup & q_a\\
%   0 &  0 & 0 & 0 & \sqcup & q_a &&&
% \end{smallmatrix}
% $
\end{tabular}
\renewcommand{\arraystretch}{1}

\item [] accepted
  
\item Example: input 000

\renewcommand{\arraystretch}{1.7}  
  \begin{tabular}{llll}
$ q_0
\begin{smallmatrix}
  0 &0 & 0 \\
  \sqcup & \sqcup & \sqcup 
\end{smallmatrix}
$
&
$
\begin{smallmatrix}
  0 \\
  \sqcup 
\end{smallmatrix}
q_1
\begin{smallmatrix}
  0 & 0  \\
  \sqcup & \sqcup 
\end{smallmatrix}
$
&           
$\cdots
$
&
$
\begin{smallmatrix}
  0 & 0 & 0 \\
  \sqcup & 0  & 0 
\end{smallmatrix}
q_1
\begin{smallmatrix}
  \sqcup\\
  \sqcup 
\end{smallmatrix}
$
\\
$
\begin{smallmatrix}
  0 & 0  \\
  \sqcup  & 0 
\end{smallmatrix}
q_2
\begin{smallmatrix}
  0\\
  0
\end{smallmatrix}
$
&
$
\begin{smallmatrix}
  0 & & 0 & 0 & q_3\\
  \sqcup & q_3 & 0 & 0 &
\end{smallmatrix}
$
& 
$
\begin{smallmatrix}
  & 0  &  0 & q_2 & 0\\
  q_2 & \sqcup & 0 &  & 0 
\end{smallmatrix}
$
&
\end{tabular}
\renewcommand{\arraystretch}{1}

\item [] rejected

\end{itemize}

\end{frame}

\end{document}

%%% Local Variables:
%%% mode: latex
%%% TeX-master: t
%%% End:

