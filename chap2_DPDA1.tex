\input{header}

\AtBeginSubsection[]
{
	\begin{frame}<beamer>
		\frametitle{Outline}
		\tableofcontents[current,currentsubsection]
	\end{frame}
}

\begin{document}

\begin{frame}[allowframebreaks] \frametitle{Deterministic CFL}
  \begin{itemize}
  \item Recall that PDA is \alert{non-deterministic}
  \item We can actually define \alert{deterministic} PDA (DPDA)
    
\item In Chapter 1,
  \begin{center}
    DFA $\equiv$ NFA
  \end{center}
\item [] Both generate regular languages
\item But 
  \begin{center}
    PDA $\neq$ DPDA
  \end{center}
and therefore
\begin{center}
  CFL $\neq$ DCFL
\end{center}
\item DPDA was not discussed in earlier versions of the textbook
\item As this topic is less important, we will only explain what DPDA is
  and will not get into more details
\item It's more complicated to define DPDA than PDA
\item The reason is that in DPDA we must ensure the deterministic moves
\end{itemize}\end{frame} \begin{frame}[allowframebreaks] \frametitle{Formal definition of DPDA}
  \begin{itemize}
\item $(Q,\Sigma, \Gamma, \delta, q_0, F)$

\item [] $Q, \Sigma, \Gamma, F$: finite sets
\begin{enumerate}
\item $Q$: states
\item $\Sigma$: alphabet
\item $\Gamma$: stack alphabet
\item $\delta$: 
  \begin{equation*}
Q \times 
\Sigma_{\epsilon} \times \Gamma_{\epsilon}
\rightarrow (Q\times \Gamma_\epsilon)\cup \{\emptyset\} 
\end{equation*}
\item $q_0 \in Q$: start state
\item $F \subset Q$: set of accept states
\end{enumerate}
\item Note for PDA
  \begin{equation*}
\delta: Q \times 
\Sigma_{\epsilon} \times \Gamma_{\epsilon}
\rightarrow P(Q\times \Gamma_\epsilon)
\end{equation*}
\item Also $\delta$ satisfies $\forall q \in Q, a \in \Sigma, x \in
\Gamma$, exactly one of
\begin{equation*}
  \delta(q,a,x), \qquad \delta(q,a,\epsilon), \qquad
\delta(q, \epsilon, x), \qquad \delta(q, \epsilon, \epsilon)
\end{equation*}
is not $\emptyset$

\item Reason: at $q$ all four can be taken at PDA
\item Rule: follow the one which is not $\emptyset$
\end{itemize}\end{frame} \begin{frame}[allowframebreaks] \frametitle{Acceptance and rejection of DPDA}
  \begin{itemize}
\item Acceptance: same as DFA. 
\item [] Reach an accept state after the last symbol
\item [] Otherwise: reject

\item Rejection: occurs if
  \begin{enumerate}
  \item not at an accept state after the last symbol (same as DFA)
  \item DPDA fails to read the input
    \begin{enumerate}
    \item pop an empty stack
    \item endless $\epsilon$-input moves
    \end{enumerate}
  \end{enumerate}

\item Example: pop an empty stack

  \begin{center}
  \begin{tabular}{c|cccc}
    & \multicolumn{2}{c}{0} & \multicolumn{2}{c}{$\epsilon$}\\
& 0 & $\epsilon$ & 0 & $\epsilon$ \\ \hline
q & $\emptyset$ & $\emptyset$ & (q, $\epsilon$) & $\emptyset$
  \end{tabular}
\end{center}

input 0,$\emptyset$ is rejected: the only possible move is to pop up
zero, but the stack is empty

\item Example: fails to read the whole string

  \begin{center}
  \begin{tabular}{c|cccc}
    & \multicolumn{2}{c}{0} & \multicolumn{2}{c}{$\epsilon$}\\
& 0 & $\epsilon$ & 0 & $\epsilon$
\\ \hline
q & $\emptyset$ & $\emptyset$ & $\emptyset$& (q, $1$)
  \end{tabular}
\end{center}
input $0,\emptyset$ is rejected: endless $\epsilon$-input
\begin{equation*}
  0,\emptyset \rightarrow 0, \{1\}, \rightarrow 0, \{1, 1\}
  \rightarrow \cdots
\end{equation*}
\end{itemize}\end{frame}

\end{document}
