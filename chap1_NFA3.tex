\input{header}

\AtBeginSection[]
{
	\begin{frame}<beamer>
		\frametitle{Outline}
		\tableofcontents[current,currentsubsection]
	\end{frame}
}

\begin{document}

\begin{frame}[allowframebreaks] \frametitle{DFA $\equiv$ NFA}
  \begin{itemize}
\item DFA $\Rightarrow$ NFA

\item Formally, a language recognized by a DFA $\rightarrow$
  recognized by an NFA

\item The proof is easy because a DFA is an NFA
\item However, \alert{formally DFA is not an NFA}
because DFA uses $\Sigma$ but not $\Sigma_\epsilon$

\item [] Can easily handle this by adding
  \begin{equation*}
  q_i, \epsilon \rightarrow \emptyset
\end{equation*}
\item The other direction: NFA $\Rightarrow $ DFA

\item Need to convert NFA to an equivalent DFA
\item [] That is, they recognize the same language
\item We do the proof by an example
\end{itemize}\end{frame}

\begin{frame}[allowframebreaks] \frametitle{Example 1.41}
  \begin{itemize}
  \item Consider the following NFA (we discussed this NFA before)
    \begin{center}
\begin{tikzpicture}
\node[state, initial, accepting] (q1) {$q_1$};
\node[state, below left of=q1] (q2) {$q_2$};
\node[state, below right of=q1] (q3) {$q_3$};

\draw (q1) edge[left] node{$b$} (q2)
(q2) edge[loop left, left] node{$a$} (q2)
(q1) edge[bend right, left] node{$\epsilon$} (q3)
(q2) edge[below] node{$a, b$} (q3)
(q3) edge[bend right, right] node{$a$} (q1);
\end{tikzpicture}
\end{center}
\item The resulting DFA diagram
\end{itemize}
\begin{center}
\begin{tikzpicture}
\node[state] (phi) {$\emptyset$};
\node[state, accepting, right of=phi] (1) {$\{1\}$};
\node[state, right of=1] (2) {$\{2\}$};
\node[state, accepting, right of=2] (12) {$\{1, 2\}$};
\node[state, below of=phi] (3) {$\{3\}$};
\node[state, initial, initial where=above, accepting, right of=3] (13) {$\{1, 3\}$};
\node[state, right of=13] (23) {$\{2, 3\}$};
\node[state, accepting, right of=23] (123) {$\{1, 2, 3\}$};
\draw (phi) edge[loop left] node{$a, b$} (phi)
(1)
edge[above] node{$a$} (phi)
(1)
edge[above] node{$b$} (2)
(2)
edge[right] node{$a$} (23)
(2)
edge[above] node{$b$} (3)
(12) edge[above, pos=.3, left=2pt] node{$a, b$} (23)
(3)
edge[left] node{$b$} (phi)
(3)
edge[below] node{$a$} (13)
(13) edge[loop right,left] node{$a$} (13)
(13) edge[above] node{$b$} (2)
(23) edge[bend left, above] node{$a$} (123)
(23) edge[bend left, below] node{$b$} (3)
(123) edge[loop below, above] node{$a$} (123)
(123) edge[bend left, below] node{$b$} (23);
\end{tikzpicture}
\end{center}

\end{frame}

\begin{frame}[allowframebreaks]
  \frametitle{Explanation of the Procedure}
\begin{itemize}
\item Each state is a subset of $\{1,2,3\}$
\item Let's check details of

  \begin{center}
    \begin{tikzpicture}
      \node[state] (12) {$\{1, 2\}$};
      \node[state, right of=12] (23) {$\{2, 3\}$};
\draw (12) edge[above] node{$a$} (23);      
    \end{tikzpicture}
  \end{center}

\item We see
  \begin{equation*}
    \begin{split}
      & q_1 \xrightarrow{a} \emptyset\\
& q_2 \xrightarrow{a} \{q_2, q_3\}      
    \end{split}
  \end{equation*}
Thus
\begin{equation*}
  \emptyset \cup \{ 2, 3\} = \{2, 3\}
\end{equation*}
\item Start state:
  \begin{center}
    $\{1, 3\}$ but not $\{1\}$
  \end{center}
  The reason is that in the beginning, even without any
  input, we can already reach $q_3$
\item Accept states: any state including $q_1$ is an accept state
\end{itemize}
\end{frame}
\begin{frame}[allowframebreaks]
  \frametitle{Removing unused states}
  \begin{itemize}
  \item Some states can never be reached
  \item We can remove them to simplify the diagram
  \item Turns out any state \alert{having 1 but without 3}
    can never be reached
  \end{itemize}
  
\begin{center}
\begin{tikzpicture}
\node[state, initial, initial where=left, accepting, right of=3] (13) {$\{1, 3\}$};
\node[state, below of=13] (2) {$\{2\}$};
\node[state, right of=13] (3) {$\{3\}$};
\node[state, right of=3] (phi) {$\emptyset$};
\node[state, right of=2] (23) {$\{2, 3\}$};
\node[state, accepting, right of=23] (123) {$\{1, 2, 3\}$};
\draw
(phi) edge[loop above] node{$a, b$} (phi)
(2)
edge[above] node{$a$} (23)
(2)
edge[above] node{$b$} (3)
(3)
edge[above] node{$b$} (phi)
(3)
edge[above] node{$a$} (13)
(13) edge[loop above,left] node{$a$} (13)
(13) edge[left] node{$b$} (2)
(23) edge[bend left, above] node{$a$} (123)
(23) edge[right] node{$b$} (3)
(123) edge[loop right, below] node{$a$} (123)
(123) edge[bend left, below] node{$b$} (23);
\end{tikzpicture}
\end{center}
\end{frame}


\begin{frame}[allowframebreaks] \frametitle{More explanation
    of example 1.41}
\begin{itemize}

\item Idea: 

  
\item [] Each node includes all states at the current layer
  
\item Example: $baa$


  \begin{center}
\begin{tikzpicture} [level distance=30pt]% grow=right
  \node  {$q_1$}
  child [grow=right] {node  {$q_3$} edge from parent[draw=none]}
  child [grow=down] {node  {$q_2$}
    child {node {$q_2$}
      child {node  {$q_2$}
        child [grow=left] {node (4) {} edge from parent[draw=none]
          child [grow=up] {node (3) {} edge from parent[draw=none]
            child [grow=up] {node (2) {} edge from parent[draw=none]
              child [grow=up] {node (1) {} edge from parent[draw=none]
              }
            }
          }
        }
      }
      child {node {$q_3$}}
    }
    child {edge from parent[draw=none]}
    child {node {$q_3$}
      child {node {$q_1$}}
      child {node {$q_3$}}      
    }    
  };
  \path (1) -- (2) node [midway] {b};
  \path (2) -- (3) node [midway] {a};
  \path (3) -- (4) node [midway] {a};
\end{tikzpicture}
\end{center}
We see
\begin{equation*}
  \{1\} \xrightarrow{b} \{2\} \xrightarrow{a} \{2, 3\}
  \xrightarrow{a} \{1, 2, 3\}
\end{equation*}
\item Proof
  \item[] Given NFA
    \begin{equation*}
(Q,\Sigma, \delta, q_0, F)
\end{equation*}
We would like to convert it to a DFA
\begin{equation*}
(Q', \Sigma, \delta', q'_0, F')
\end{equation*}
Details of this DFA:
\begin{itemize}
\item $Q'=P(Q)$
\item $q_0' \in P(Q)$ includes
  \begin{center}
 $q_0$ + states
reached by $\epsilon$
\end{center}
We call such a set $E(q_0)$
\item $F'=
\{R\mid R \in Q', R \cap F\neq \emptyset\}$
\item $\delta'$:
  \begin{equation*}
    \delta'(R,a)=
\cup_{r\in R} E(\delta(r,a))
\end{equation*}
Note that we cannot just do
\begin{equation*}
\cup_{r\in R}\delta(r,a)
\end{equation*}
\end{itemize}
\end{itemize}
\end{frame}

\end{document}
