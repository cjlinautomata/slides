\input{header}

\AtBeginSubsection[]
{
	\begin{frame}<beamer>
		\frametitle{Outline}
		\tableofcontents[current,currentsubsection]
	\end{frame}
}

\begin{document}

\begin{frame}[allowframebreaks] \frametitle{Deterministic CFL}
  \begin{itemize}
  \item Recall that PDA is \alert{non-deterministic}
  \item We can actually define \alert{deterministic} PDA (DPDA)
    
\item In Chapter 1,
  \begin{center}
    DFA $\equiv$ NFA
  \end{center}
\item [] Both generate regular languages
\item But 
  \begin{center}
    PDA $\neq$ DPDA
  \end{center}
and therefore
\begin{center}
  CFL $\neq$ DCFL
\end{center}
\item We will only explain what DPDA is

\end{itemize}\end{frame} \begin{frame}[allowframebreaks] \frametitle{Formal definition of DPDA}
  \begin{itemize}
\item $(Q,\Sigma, \Gamma, \delta, q_0, F)$

\item [] $Q, \Sigma, \Gamma, F$: finite sets
\begin{enumerate}
\item $Q$: states
\item $\Sigma$: alphabet
\item $\Gamma$: stack alphabet
\item $\delta$: 
  \begin{equation*}
Q \times 
\Sigma_{\epsilon} \times \Gamma_{\epsilon}
\rightarrow (Q\times \Gamma_\epsilon)\cup \{\emptyset\} 
\end{equation*}
\item $q_0 \in Q$: start state
\item $F \subset Q$: set of accept states
\end{enumerate}
\item Note for PDA
  \begin{equation*}
\delta: Q \times 
\Sigma_{\epsilon} \times \Gamma_{\epsilon}
\rightarrow P(Q\times \Gamma_\epsilon)
\end{equation*}
\item Also $\delta$ satisfies $\forall q \in Q, a \in \Sigma, x \in
\Gamma$, exactly one of
\begin{equation*}
  \delta(q,a,x), \qquad \delta(q,a,\epsilon), \qquad
\delta(q, \epsilon, x), \qquad \delta(q, \epsilon, \epsilon)
\end{equation*}
is not $\emptyset$

\item Reason: at $q$ all four can be taken at PDA
\item Rule: follow the one which is not $\emptyset$
\item If stack is empty, DPDA can move only if
  \begin{equation}
    \label{eq:cannotmove}
    \delta(q, ?, \epsilon) \neq \emptyset
  \end{equation}
Otherwise, reject without reading rest of input. 

\end{itemize}\end{frame} \begin{frame}[allowframebreaks] \frametitle{Acceptance and rejection of DPDA}
  \begin{itemize}
\item Acceptance: same as DFA. 
\item [] Reach an accept state after the last symbol
\item [] Otherwise: reject

\item Rejection: occurs if
  \begin{enumerate}
  \item not at an accept state after the last symbol (same as DFA)
  \item DPDA fails to read the input
    \begin{enumerate}
    \item pop an empty stack
    \item endless $\epsilon$-input moves
    \end{enumerate}
  \end{enumerate}

\item Example
  \begin{equation*}
    \{ 0^n 1^n \mid n \geq 0\}
  \end{equation*}
  The diagram we had earlier is
  \begin{center}
  \begin{tikzpicture}
\node[state,initial,accepting] (q_1) {$q_1$};
\node[state] (q_2) [right of=q_1] {$q_2$};
\node[state] (q_3) [below of=q_2] {$q_{3}$};
\node[state,accepting] (q_4) [left of=q_3] {$q_{4}$};      
  \path 
  (q_1) edge[above]  node {$\epsilon, \epsilon \rightarrow \$ $} (q_2)
  (q_2) edge[loop right]  node {$0,\epsilon \rightarrow 0$} (q_2)
  (q_2) edge[right]  node {$1, 0 \rightarrow \epsilon$} (q_3)
  (q_3) edge[loop right]  node {$1, 0 \rightarrow \epsilon$} (q_3)  
  (q_3) edge[below] node {$\epsilon, \$ \rightarrow \epsilon$} (q_4);
  \end{tikzpicture}
\end{center}
  
\item The $\delta$ function
\begin{tabular}{lccc|ccc|ccc}
&
\multicolumn{3}{c|}{0} &
\multicolumn{3}{c|}{1} &
\multicolumn{3}{c}{$\epsilon$}\\ \hline
& 0 & \$ & $\epsilon$ 
& 0 & \$ & $\epsilon$ 
& 0 & \$ & $\epsilon$ \\ \hline
$q_1$ &$\emptyset$&$\emptyset$&$\emptyset$&$\emptyset$&$\emptyset$&
$\emptyset$&$\emptyset$&$\emptyset$& $(q_2,\$)$\\
$q_2$ &$\emptyset$&$\emptyset$&$(q_2,0)$&$(q_3,\epsilon)$&$q_r$&
$\emptyset$&$\emptyset$&$\emptyset$& $\emptyset$\\
$q_3$ &$q_r$&$\emptyset$&$\emptyset$&$(q_3,\epsilon)$&$\emptyset$&$\emptyset$&$\emptyset$&$(q_4,\epsilon)$&$\emptyset$ \\
$q_4$ &$q_r$&$q_r$&$\emptyset$&$q_r$&$q_r$&$\emptyset$&$\emptyset$&$\emptyset$&$\emptyset$ \\ 
$q_r$ &$q_r$&$q_r$&$\emptyset$&$q_r$&$q_r$&$\emptyset$&$\emptyset$&$\emptyset$&$\emptyset$ \\ 
\end{tabular}
\item For example,
  \begin{equation*}
    \delta(q_1, \epsilon, \epsilon) = q_2, \$
  \end{equation*}
  implies that
  \begin{equation*}
    \delta(q_1, 0, 1)
    \delta(q_1, 0, \epsilon)
\delta(q_1, \epsilon, 1)    = \emptyset
\end{equation*}
Thus we see everything else in the first row is $\emptyset$
\item $$q_r$$ is another state. We need it to ensure that one of the four 
is not $\emptyset$

\item Consider an input string 011
\begin{equation*}
  q_1 \mydef{\epsilon}{\rightarrow} q_2, \{\$\}
  \mydef{0}{\rightarrow} q_2, \{0\$\}
\mydef{1}{\rightarrow} q_3, \{\$\}
\mydef{\epsilon}{\rightarrow} q_4, \emptyset
\end{equation*}
\item From $q_4$, we don't know where to go
\item [] Endless $\epsilon$-input moves
  \begin{equation*}
  q_4, \emptyset \mydef{\epsilon}{\rightarrow} \emptyset
\end{equation*}
Cannot move; see \eqref{eq:cannotmove}
\begin{equation*}
  q_4, \emptyset \mydef{1}{\rightarrow} \emptyset
\end{equation*}
\item Example: pop an empty stack

  \begin{center}
  \begin{tabular}{c|cccc}
    & \multicolumn{2}{c}{0} & \multicolumn{2}{c}{$\epsilon$}\\
& 0 & $\epsilon$ & 0 & $\epsilon$ \\ \hline
a & $\emptyset$ & $\emptyset$ & (a, $\epsilon$) & $\emptyset$
  \end{tabular}
\end{center}

input 0$\emptyset$ is rejected: the only move is to pop up
zero, but the stack is empty

\item Example: that fails to read the whole string

  \begin{center}
  \begin{tabular}{c|cccc}
    & \multicolumn{2}{c}{0} & \multicolumn{2}{c}{$\epsilon$}\\
& 0 & $\epsilon$ & 0 & $\epsilon$
\\ \hline
a & $\emptyset$ & $\emptyset$ & $\emptyset$& (a, $\epsilon$)
  \end{tabular}
\end{center}
input 0 is rejected: endless $\epsilon$-input

\item We omit all the rest details

\end{itemize}\end{frame}

\end{document}
