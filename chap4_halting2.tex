\input{header}

\AtBeginSubsection[]
{
	\begin{frame}<beamer>
		\frametitle{Outline}
		\tableofcontents[current,currentsubsection]
	\end{frame}
}

\begin{document}

\begin{frame}[allowframebreaks] \frametitle{Halting problem undecidable}
  \begin{itemize}
\item Recall the halting problem is
  \begin{equation*}
A_{TM}
=\{(M,w)\mid M: \mbox{TM, accepts } w\}
\end{equation*}
We prove it is undecidable
by contradiction
\item Assume there is an $H$ that is a decider for $A_{TM}$
\item [] Then $H$ satisfies
  \begin{equation*}
    H(\langle  M,w\rangle )=
    \begin{cases}
      \mbox{accept} & \mbox{ if } M \mbox{ accepts } w\\
\mbox{reject} & \mbox{ otherwise}
    \end{cases}
  \end{equation*}
\item Construct a new TM $D$ with $H$ as a subroutine

\item For $D$, the input is $\langle  M\rangle $, where
  $M$ is a TM

\item [] It runs $H$ on $\langle  M,\langle  M\rangle \rangle $
and 
outputs the opposite result of $H$

% \item $M$ can be an input of $M$

% A C compiler can be written in C

% Do not worry how this compiler is compiled

\item The machine $D$ satisfies
  \begin{equation*}
    D(\langle  M\rangle )
=
\begin{cases}
  \mbox{accept } & 
  \mbox{if } M \mbox{ rejects } \langle  M\rangle \\
\mbox{if } M \mbox{ accepts } \langle  M\rangle
\end{cases}
  \end{equation*}

\item But we get a contradiction
  \begin{equation*}
    D(\langle  D\rangle )
=
\begin{cases}
  \mbox{accept } & 
\mbox{if } D \mbox{ rejects } \langle  D\rangle \\
\mbox{reject} & \mbox{if } D \mbox{ accepts } \langle  D\rangle 
\end{cases}
  \end{equation*}
\item We said earlier that the diagonalization method
  is used for the proof. Is that the case?
\item We show that indeed it is used
  
\end{itemize}\end{frame} \begin{frame}[allowframebreaks] \frametitle{Diagonalization in the proof}
  \begin{itemize}
\item Set of TMs is countable so we can have
  \begin{center}
    \begin{tabular}{c|ccc}
& $\langle  M_1\rangle $ & $\langle  M_2\rangle $ & $\langle  M_3\rangle $\\ \hline
$M_1$ & A & &A\\
$M_2$ & A & A &A\\
$\vdots$ &&&
    \end{tabular}
  \end{center}
blank entries: unknown if reject or loop

\item But $H$ knows the solution as it is a decider
  
  \begin{center}
    \begin{tabular}{c|ccc}
& $\langle  M_1\rangle $ & $\langle  M_2\rangle $ & $\langle  M_3\rangle $\\ \hline
$M_1$ & A & R & A\\
$M_2$ & A & A & A\\
$\vdots$ &&&
    \end{tabular}
  \end{center}

\item $D$ outputs \alert{opposite of diagonal entries}
  \begin{center}
    \begin{tabular}{c|cccc}
& $\langle  M_1\rangle $ & $\langle  M_2\rangle $ & $\ldots$ & $\langle  D\rangle $\\ \hline
$M_1$ & R & &&\\
$M_2$ &  & R && \\
&& & $\ddots$ & \\
$D$ &&&& ?
    \end{tabular}
  \end{center}

\end{itemize}\end{frame} \begin{frame}[allowframebreaks] \frametitle{co-Turing-recognizable Language}
  \begin{itemize}
\item Definition: a language is co-Turing-recognizable
if its complement is Turing-recognizable
\item Theorem 4.22
\item[] Decidable $\Leftrightarrow$
Turing-recognizable and co-Turing-recognizable
\item Why complement may not be Turing-recognizable?
\item [] Note that ``recognizable'' means any
  \begin{center}
  $w \in $ language
\end{center}
is accepted by the machine in \alert{a finite steps}
\item That is, no infinite loop
\item Example:
\item [] $A_{TM}$ Turing-recognizable but not decidable

\item [] $\overline{A_{TM}}$ not Turing-recognizable
  \begin{equation*}
    w \in \overline{A_{TM}}
  \end{equation*}
  $\Rightarrow$ reject or loop


\item Why not
  \begin{equation*}
    \begin{split}
&   \text{ Turing-recognizable} \\
\Rightarrow & \text{ complement Turing-recognizable}
\end{split}
\end{equation*}
\item What if we swap $q_{accept},q_{reject}$?

\item If
  \begin{center}
  $a \notin A$ and loop occurs
\end{center}
then
\begin{center}
$a \in \overline{A}$, but TM still loops
\end{center}
\item Recall that $\overline{A}$ is Turing-recognizable means
that any
$a \in \overline{A}$ can be recognized in a finite \#
of steps.

\item Proof of Theorem 4.22
\item ``$\Rightarrow$''

\item [] Decidable $\Rightarrow$ Turing-recognizable

\item [] Complement $\Rightarrow$ decidable $\Rightarrow$
Turing-recognizable

\item ``$\Leftarrow$'' Now  $A,\overline{A}$ are Turing-recognizable
  by two machines $M_1, M_2$
\item Construct a new machine $M$: for any input $w$
\begin{enumerate}
\item Run $M_1,M_2$ in parallel
\item $M_1$ accept $\Rightarrow$ accept, 
$M_2$ accept $\Rightarrow$ reject
\end{enumerate}
\item Never infinity loop

\item $M$ accepts all strings in $A$, reject all not in $A$

\item Thus $A$ is decidable with a decider $M$
\end{itemize}\end{frame}
\end{document}

%%% Local Variables:
%%% mode: latex
%%% TeX-master: t
%%% End:
