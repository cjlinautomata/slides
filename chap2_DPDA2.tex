\input{header}

\AtBeginSubsection[]
{
	\begin{frame}<beamer>
		\frametitle{Outline}
		\tableofcontents[current,currentsubsection]
	\end{frame}
}

\begin{document}

\begin{frame}[allowframebreaks] \frametitle{Example of DPDA}
  \begin{itemize}
\item Example
  \begin{equation*}
    \{ 0^n 1^n \mid n \geq 0\}
  \end{equation*}
  The diagram we had earlier is
  \begin{center}
  \begin{tikzpicture}
\node[state,initial,accepting] (q_1) {$q_1$};
\node[state] (q_2) [right of=q_1] {$q_2$};
\node[state] (q_3) [below of=q_2] {$q_{3}$};
\node[state,accepting] (q_4) [left of=q_3] {$q_{4}$};      
  \path 
  (q_1) edge[above]  node {$\epsilon, \epsilon \rightarrow \$ $} (q_2)
  (q_2) edge[loop right]  node {$0,\epsilon \rightarrow 0$} (q_2)
  (q_2) edge[right]  node {$1, 0 \rightarrow \epsilon$} (q_3)
  (q_3) edge[loop right]  node {$1, 0 \rightarrow \epsilon$} (q_3)  
  (q_3) edge[below] node {$\epsilon, \$ \rightarrow \epsilon$} (q_4);
  \end{tikzpicture}
\end{center}
  
\item The $\delta$ function
\begin{tabular}{lccc|ccc|ccc}
&
\multicolumn{3}{c|}{0} &
\multicolumn{3}{c|}{1} &
\multicolumn{3}{c}{$\epsilon$}\\ \hline
& 0 & \$ & $\epsilon$ 
& 0 & \$ & $\epsilon$ 
& 0 & \$ & $\epsilon$ \\ \hline
$q_1$ &$\emptyset$&$\emptyset$&$\emptyset$&$\emptyset$&$\emptyset$&
$\emptyset$&$\emptyset$&$\emptyset$& $(q_2,\$)$\\
$q_2$ &$\emptyset$&$\emptyset$&$(q_2,0)$&$(q_3,\epsilon)$&$q_r$&
$\emptyset$&$\emptyset$&$\emptyset$& $\emptyset$\\
$q_3$ &$q_r$&$\emptyset$&$\emptyset$&$(q_3,\epsilon)$&$\emptyset$&$\emptyset$&$\emptyset$&$(q_4,\epsilon)$&$\emptyset$ \\
$q_4$ &$q_r$&$q_r$&$\emptyset$&$q_r$&$q_r$&$\emptyset$&$\emptyset$&$\emptyset$&$\emptyset$ \\ 
$q_r$ &$q_r$&$q_r$&$\emptyset$&$q_r$&$q_r$&$\emptyset$&$\emptyset$&$\emptyset$&$\emptyset$ \\ 
\end{tabular}
\item For the first row,
  \begin{equation*}
    \delta(q_1, \epsilon, \epsilon) = (q_2, \$)
  \end{equation*}
  implies that
  \begin{equation*}
    \delta(q_1, a, x) =
    \delta(q_1, a, \epsilon)=
\delta(q_1, \epsilon, x)    = \emptyset, \forall a \in \Sigma, x \in \Gamma
\end{equation*}
Thus we see everything else in the first row is $\emptyset$
\item For the second row,
  \begin{equation*}
    \delta(q_2, 0, \epsilon) = (q_2, 0)
  \end{equation*}
  implies
  \begin{center}
\begin{tabular}{lccc|ccc|ccc}
&
\multicolumn{3}{c|}{0} &
\multicolumn{3}{c|}{1} &
\multicolumn{3}{c}{$\epsilon$}\\ \hline
& 0 & \$ & $\epsilon$ 
& 0 & \$ & $\epsilon$ 
& 0 & \$ & $\epsilon$ \\ \hline
$q_2$ &$\emptyset$&$\emptyset$&$(q_2,0)$&&&
&$\emptyset$&$\emptyset$& $\emptyset$\\
\end{tabular}
\end{center}
Then
\begin{equation*}
  \delta(q_2, 1, 0) = (q_3, \epsilon)
\end{equation*}
further implies
  \begin{center}
\begin{tabular}{lccc|ccc|ccc}
&
\multicolumn{3}{c|}{0} &
\multicolumn{3}{c|}{1} &
\multicolumn{3}{c}{$\epsilon$}\\ \hline
& 0 & \$ & $\epsilon$ 
& 0 & \$ & $\epsilon$ 
& 0 & \$ & $\epsilon$ \\ \hline
$q_2$ &$\emptyset$&$\emptyset$&$(q_2,0)$&$(q_3, \epsilon)$&&$\emptyset$
&$\emptyset$&$\emptyset$& $\emptyset$\\
\end{tabular}
\end{center}
Finally, we must have

  \begin{center}
\begin{tabular}{lccc|ccc|ccc}
&
\multicolumn{3}{c|}{0} &
\multicolumn{3}{c|}{1} &
\multicolumn{3}{c}{$\epsilon$}\\ \hline
& 0 & \$ & $\epsilon$ 
& 0 & \$ & $\epsilon$ 
& 0 & \$ & $\epsilon$ \\ \hline
$q_2$ &$\emptyset$&$\emptyset$&$(q_2,0)$&$(q_3, \epsilon)$&\alert{$\neq \emptyset$}&$\emptyset$
&$\emptyset$&$\emptyset$& $\emptyset$\\
\end{tabular}
\end{center}
because
\begin{equation*}
  \delta(q_2, 1, \epsilon)
  =   \delta(q_2, \epsilon, \$)
  =   \delta(q_2, \epsilon, \epsilon) = \emptyset
\end{equation*}
\item Thus we consider an additional state $q_r$. We need it to ensure that one of the four 
is not $\emptyset$
\item Formally, we should have
  \begin{equation*}
      \delta(q_2, 1, \$) = (q_r, \epsilon)
  \end{equation*}
though we wrote only $q_r$ for simplicity
\item For the third row, we have
  \begin{center}
\begin{tabular}{lccc|ccc|ccc}
&
\multicolumn{3}{c|}{0} &
\multicolumn{3}{c|}{1} &
\multicolumn{3}{c}{$\epsilon$}\\ \hline
& 0 & \$ & $\epsilon$ 
& 0 & \$ & $\epsilon$ 
& 0 & \$ & $\epsilon$ \\ \hline
$q_3$ &&&&$(q_3,\epsilon)$&&$\emptyset$&$\emptyset$&&$\emptyset$ \\
\end{tabular}
\end{center}
Then
\begin{center}
  \begin{tabular}{lccc|ccc|ccc}
&
\multicolumn{3}{c|}{0} &
\multicolumn{3}{c|}{1} &
\multicolumn{3}{c}{$\epsilon$}\\ \hline
& 0 & \$ & $\epsilon$ 
& 0 & \$ & $\epsilon$ 
& 0 & \$ & $\epsilon$ \\ \hline
$q_3$ &&$\emptyset$&$\emptyset$&$(q_3,\epsilon)$&$\emptyset$&$\emptyset$&$\emptyset$&$(q_4,\epsilon)$&$\emptyset$ \\
\end{tabular}
\end{center}
In the end
\begin{equation*}
  \delta(q_3, 0, 0) = q_r
\end{equation*}
\item For the fourth row, there is no out link at $q_4$
\item We let
  \begin{equation*}
    \delta(q_4, a, x) = q_r
  \end{equation*}
and
\begin{equation*}
  \delta(q_4, a, \epsilon) =
  \delta(q_4, \epsilon, x) =
  \delta(q_4, \epsilon, \epsilon) = \emptyset
\end{equation*}
\item We can do the same for the last row
\item Consider an input string 011
\begin{equation*}
  q_1 \mydef{\epsilon}{\rightarrow} q_2, \{\$\}
  \mydef{0}{\rightarrow} q_2, \{0\$\}
\mydef{1}{\rightarrow} q_3, \{\$\}
\mydef{\epsilon}{\rightarrow} q_4, \emptyset
\end{equation*}
\item From $q_4$, the two possible moves are
  \begin{equation*}
    \delta(q_4, 1, \epsilon) \text{ and } \delta(q_4, \epsilon, \epsilon)
  \end{equation*}
Both are $\emptyset$, so  we don't know where to go
\item Thus this input string is rejected
\end{itemize}\end{frame}

\end{document}
