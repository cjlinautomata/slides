\input{header}

\AtBeginSection[]
{
	\begin{frame}<beamer>
		\frametitle{Outline}
		\tableofcontents[current,currentsubsection]
	\end{frame}
}

\begin{document}

\begin{frame}[allowframebreaks] \frametitle{Nondeterminism}
  \begin{itemize}
\item Deterministic algorithm:

\item [] Given current state and current input, next step is known
\item Nondeterministic algorithm:

\item [] Several choices are possible
  
\item They will be respectively called
  \begin{center}
  DFA: deterministic finite automata
\end{center}
  and
  NFA: non-deterministic finite automata
\item Fig 1.27

\begin{center}
    \begin{tikzpicture}
\node[state,initial] (q_1) {$q_1$};
\node[state] (q_2) [right of=q_1] {$q_2$};
\node[state] (q_3) [right of=q_2] {$q_{3}$};
\node[state,accepting] (q_4) [right of=q_3] {$q_{4}$};      
  \path 
        (q_1) edge[loop above] node {$0,1$} (q_1)
  (q_1) edge[above]  node {$1$} (q_2)
  (q_2) edge[above]  node {$0,\epsilon$} (q_3)
  (q_3) edge[above]  node {$1$} (q_4)
  (q_4) edge[loop below] node {$0,1$} (q_4);
      \end{tikzpicture}
    \end{center}
  
  \item $\delta$ is not a function any more: $\delta(q_1, 1) = q_1
$ or $q_2$

\item $\epsilon$ between $q_2$ and $q_3$: $q_2$ can move to
$q_3$ without any input
\item How to run a string? It can be run by split

\item A kind of parallel machines
\item ex: 010110

  Fig 1.29

\begin{tikzpicture} [level distance=30pt]% grow=right
\node  {$q_1$}
  child {node  {$q_1$}
    child {node {$q_1$}
      child {node  {$q_1$}
        child {node  {$q_1$}
          child {node  {$q_1$}
            child {node  {$q_1$}
              child [grow=left] {node (7) {} edge from parent[draw=none]
                child [grow=up] {node (6) {} edge from parent[draw=none]
                  child [grow=up] {node (5) {} edge from parent[draw=none]
                    child [grow=up] {node (4) {} edge from parent[draw=none]
                      child [grow=up] {node (3) {} edge from parent[draw=none]
                        child [grow=up] {node (2) {} edge from parent[draw=none]
                          child [grow=up] {node (1) {} edge from parent[draw=none]
                          }
                        }
                      }
                    }
                  }
                }
              }
            }
          }
          child {node  {$q_2$}
            child {node  {$q_3$}}
          }
          child {node  {$q_3$}}
        }
        child {node  {$q_2$}}
        child {node  {$q_3$}
          child {node  {$q_4$}
            child {node  {$q_4$}
            }          
          }
        }
      }
    }
    child {node  {$q_2$}
      child {node  {$q_3$}
        child {edge from parent[draw=none]}
        child {edge from parent[draw=none]}        
        child {node  {$q_4$}
          child {node  {$q_4$}
            child {node  {$q_4$}
            }
          }
        }
      }
    }
    child {node  {$q_3$}}
  };
  \path (1) -- (2) node [midway] {0};
  \path (2) -- (3) node [midway] {1};
  \path (3) -- (4) node [midway] {0};
  \path (4) -- (5) node [midway] {1};
  \path (5) -- (6) node [midway] {1};
  \path (6) -- (7) node [midway] {0};
\end{tikzpicture}

\end{itemize}\end{frame} \begin{frame}[allowframebreaks] \frametitle{Some NFA examples}
  \begin{itemize}
\item Example 1.30

\item [] Strings with 1 in 3rd position from the end

\item [] 00100, 0100 are accepted , but 0010 is not


\item [] Fig 1.31

\begin{center}
    \begin{tikzpicture}
\node[state,initial] (q_1) {$q_1$};
\node[state] (q_2) [right of=q_1] {$q_2$};
\node[state] (q_3) [right of=q_2] {$q_{3}$};
\node[state,accepting] (q_4) [right of=q_3] {$q_{4}$};      
  \path 
        (q_1) edge[loop above] node {$0,1$} (q_1)
  (q_1) edge[above]  node {$1$} (q_2)
  (q_2) edge[above]  node {$0,1$} (q_3)
  (q_3) edge[above]  node {$0,1$} (q_4);
      \end{tikzpicture}
    \end{center}

\item DFA and NFA

\item They are equivalent. We will explain this later

\item [] Fig 1.32.  For this example in fact we are able to design a DFA
  for this language

\end{itemize}
\begin{tikzpicture}
\node[state, initial] (000) {$q_{000}$};
\node[state, accepting, right of=000] (100) {$q_{100}$};
\node[state, right of=100] (010) {$q_{010}$};
\node[state, accepting, right of=010] (110) {$q_{110}$};
\node[state, below of=000] (001) {$q_{001}$};
\node[state, accepting, right of=001] (101) {$q_{101}$};
\node[state, right of=101] (011) {$q_{011}$};
\node[state, accepting, right of=011] (111) {$q_{111}$};

\draw (000) edge[loop above] node{$0$} (000)
(000) edge[left] node{$1$} (001)
(100) edge[above] node{$0$} (000)
(100) edge[above] node{$1$} (001)

(010) edge[bend right=10,above] node{$1$} (101)
(010) edge[above] node{$0$} (100)

(110) edge[bend right,above] node{$0$} (100)
(110) edge[above] node{$1$} (101)

(001) edge[bend right,below] node{$1$} (011)
(001) edge[above] node{$0$} (010)

(101) edge[bend right=10,below] node{$0$} (010)
(101) edge[above] node{$1$} (011)

(011) edge[above] node{$1$} (111)
(011) edge[above] node{$0$} (110)

(111) edge[loop right, below] node{$1$} (111)
(111) edge[right] node{$0$} (110);

\end{tikzpicture}

\begin{itemize}
\item 
Idea of this diagram: using 8 states
to record the past 3 digits so far

\item The idea is simple. But 
why can we use 000 as the start state? 
\item Looks like we need other nodes:
  \begin{equation*}
    \_\_\_,
    \_\_0,
    \_\_1,
    \_01,
    \_10,
    \_00,
    \_11 
\end{equation*}
\item Then we see that the path is the same as if
  we start from  000
\item For example,
  \begin{equation*}
    \_\_\_ \xrightarrow{0}
    \_\_0 \xrightarrow{1}
    \_10
  \end{equation*}

\item A modification of the NFA 
  \begin{equation*}
    \begin{split}
& q_2 \rightarrow q_3:0,1 \Rightarrow 0,1,\epsilon\\
& q_3 \rightarrow q_4:0,1 \Rightarrow 0,1,\epsilon
\end{split}
\end{equation*}
\framebreak
\begin{center}
    \begin{tikzpicture}
\node[state,initial] (q_1) {$q_1$};
\node[state] (q_2) [right of=q_1] {$q_2$};
\node[state] (q_3) [right of=q_2] {$q_{3}$};
\node[state,accepting] (q_4) [right of=q_3] {$q_{4}$};      
  \path 
        (q_1) edge[loop above] node {$0,1$} (q_1)
  (q_1) edge[above]  node {$1$} (q_2)
  (q_2) edge[above]  node {$0,1,\epsilon$} (q_3)
  (q_3) edge[above]  node {$0,1,\epsilon$} (q_4);
      \end{tikzpicture}
    \end{center}

  \item What is the language: at least one of the last three characters
    is 1
\item How about DFA for this language?

\item [] Except $q_{000}$, all others are in $F$
\end{itemize}\end{frame} \begin{frame}[allowframebreaks] \frametitle{Example 1.33}
\begin{itemize}
\item Consider the following figure
    \begin{center}
\begin{tikzpicture}
\node[state, initial] (0) {};
\node[state, accepting, right of=0] (1) {};
\node[state, right of=1] (2) {};
\node[state, accepting, below of=1] (3) {};
\node[state, right of=3] (4) {};
\node[state, below right of=3] (5) {};

\draw (0) edge[above] node{$\epsilon$} (1)
(0) edge[below] node{$\epsilon$} (3)
(1) edge[bend right, below] node{$0$} (2)
(2) edge[bend right, above] node{$0$} (1)
(3) edge[above] node{$0$} (4)
(4) edge[above] node{$0$} (5)
(5) edge[above] node{$0$} (3);
\end{tikzpicture}
\end{center}
\item For this language, $\Sigma=\{0\}$.
  This is called unary alphabets

\item What is the language?
  \begin{equation*}
\{0^k\mid  k \text{ multiples of 2 or 3}\}
\end{equation*}
\end{itemize}\end{frame} \begin{frame}[allowframebreaks] \frametitle{Example 1.35}
  \begin{itemize}
  \item Fig 1.36
  \begin{center}
\begin{tikzpicture}
\node[state, initial, accepting] (q1) {$q_1$};
\node[state, below left of=q1] (q2) {$q_2$};
\node[state, below right of=q1] (q3) {$q_3$};

\draw (q1) edge[left] node{$b$} (q2)
(q2) edge[loop left, left] node{$a$} (q2)
(q1) edge[bend right, left] node{$\epsilon$} (q3)
(q2) edge[below] node{$a, b$} (q3)
(q3) edge[bend right, right] node{$a$} (q1);
\end{tikzpicture}
\end{center}
    
\item Accept

$\epsilon$, $a$, $baba$, $baa$ can be accepted
\item But babba is rejected
\item [] See the tree below

  \begin{center}
\begin{tikzpicture} [level distance=30pt]% grow=right
  \node  {$q_1$}
  child [grow=right] {node  {$q_3$} edge from parent[draw=none]}
  child [grow=down] {node  {$q_2$}
    child {node {$q_2$}
      child {node  {$q_3$}
        child [grow=left] {node (4) {} edge from parent[draw=none]
          child [grow=up] {node (3) {} edge from parent[draw=none]
            child [grow=up] {node (2) {} edge from parent[draw=none]
              child [grow=up] {node (1) {} edge from parent[draw=none]
              }
            }
          }
        }
      }
    }
  };
  \path (1) -- (2) node [midway] {b};
  \path (2) -- (3) node [midway] {a};
  \path (3) -- (4) node [midway] {b};
\end{tikzpicture}
\end{center}
\item This example is later used to illustrate the
  procedure for converting NFA to DFA
\end{itemize}\end{frame}
\end{document}
