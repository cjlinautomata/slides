\input{header}
\AtBeginSection[]
{
	\begin{frame}<beamer>
		\frametitle{Outline}
		\tableofcontents[current,currentsubsection]
	\end{frame}
}

\begin{document}

\begin{frame}[allowframebreaks] \frametitle{Regular Operations}
  \begin{itemize}
\item Regular operations can be used to study whether languages are regular or not

\item  That is, we aim to check for a given language,
  whether there are finite automata to recognize it or not
\item Three definitions

\item [] $A,B$ are given languages
\begin{itemize}
\item union
  \begin{equation*}
  A \cup B
\end{equation*}
\item concatenation
  \begin{equation*}
  A\circ B=\{xy\mid x\in A, y \in B\}
\end{equation*}
\item star:
  \begin{equation*}
  A^*=\{x_1 \cdots x_k\mid
k \geq 0, x_i \in A\}
\end{equation*}
\end{itemize}
\item If $k=0$, what do we mean
  \begin{equation*}
    x_1 \cdots x_k?
  \end{equation*}
  We define
  \begin{equation*}
     \epsilon: \text{ empty string}
   \end{equation*}
in this situation
\item Thus
  \begin{equation*}
   \epsilon \in A^*
 \end{equation*}
\item Example
  \begin{equation*}
    \begin{split}
& \Sigma =\{a, \ldots, z\}\\
& A=\{good, bad\} \\
& B=\{boy,girl\} \\
& A\circ B=
\{goodboy, \ldots \}\\
& A^*: \{\epsilon, good, bad, goodgood, \ldots\}
\end{split}
\end{equation*}
\item We say an operation $R$ is \alert{closed} if the
  following property holds
  \begin{center}
if $x\in A, y \in A$, then $xRy \in A$
\end{center}
Example: $N=\{1,2,\ldots\}$ is closed under multiplication
\item Th 1.25: regular languages are closed under the union operation

  \begin{equation*}
    \begin{split}
&    A_1, A_2 \text{ are regular languages} \\
\Rightarrow & A_1 \cup A_2 \text{ is regular}
  \end{split}
\end{equation*}

\item Proof
\item [] Assume we are given two automata
  \begin{eqnarray*}
&& M_1 =(Q_1, \Sigma, \delta_1, q_1, F_1)\\
&& M_2 =(Q_2, \Sigma, \delta_2, q_2, F_2)\\
  \end{eqnarray*}
\item Construct a new machine
  \begin{eqnarray*}
&& M =(Q, \Sigma, \delta, q, F)\\
&& Q=\{(r_1,r_2)\mid r_1 \in Q_1, r_2 \in Q_2\}\\
&& \delta((r_1,r_2),a)=(\delta_1(r_1,a), \delta_2(r_2,a))\\
&& q_0 = (q_1,q_2)\\
&& F=\{(r_1, r_2) \mid r_1 \in F_1 
\text{ or } r_2 \in F_2\}
  \end{eqnarray*}
\item Example: combining
  \begin{equation*}
    \begin{split}
      & \{ w \mid w \text{ has an odd \# 1's} \} \cup \\
      & \{ w \mid w \text{ has an odd \# 0's} \}
\end{split}
\end{equation*}
\begin{center}

    \begin{tikzpicture}
\node[state,initial] (q_e) {$q_e$};
\node[state,accepting] (q_o) [right of=q_e] {$q_o$};      

  \path (q_e) edge[bend left, above]  node {$1$} (q_o)
        (q_e) edge[loop above] node {$0$} (q_e)
        (q_o) edge[loop above] node {$0$} (q_o)
        (q_o) edge[bend left, below]  node {$1$} (q_e);
      \end{tikzpicture}

    \begin{tikzpicture}
\node[state,initial] (s_e) {$s_e$};
\node[state,accepting] (s_o) [right of=s_e] {$s_o$};      

  \path (s_e) edge[bend left, above]  node {$0$} (s_o)
        (s_e) edge[loop above] node {$1$} (s_e)
        (s_o) edge[loop above] node {$1$} (s_o)
        (s_o) edge[bend left, below]  node {$0$} (s_e);
      \end{tikzpicture}
    \end{center}
    \begin{center}
    \begin{tikzpicture}
      \node[state,initial] (qese) {$q_e,s_e$};
      \node[state] (qeso) [accepting, above right of = qese] {$q_e,s_o$};
      \node[state] (qose) [accepting, below right of = qese] {$q_o,s_e$};
\node[state,accepting] (qoso) [below right of = qeso] {$q_o,s_o$};                  
\path (qese) edge[bend left, above]  node {$0$} (qeso)
(qeso) edge[bend left, above]  node {$0$} (qese)
(qeso) edge[bend left, above]  node {$1$} (qoso)
(qoso) edge[bend left, above]  node {$1$} (qeso)
(qese) edge[bend left, above]  node {$1$} (qose)
(qose) edge[bend left, above]  node {$1$} (qese)
(qose) edge[bend left, above]  node {$0$} (qoso)
(qoso) edge[bend left, above]  node {$0$} (qose);
      \end{tikzpicture}
    \end{center}
  
\item Is this proof rigourously enough ?
\item [] A formal proof should be done by induction. But we don't
  provide it here
\item Th 1.26: closed under concatenation
  \begin{center}
    If $A, B$ are regular, then $A \circ B$ is regular
  \end{center}
\item [] But the proof is not easy
\item [] It's unclear where to break the input 

\item To easily do the proof,
  we introduce a new technique called \alert{nondeterminism}
  \end{itemize}\end{frame}

\end{document}
