\input{header}

\AtBeginSection[]
{
	\begin{frame}<beamer>
		\frametitle{Outline}
		\tableofcontents[current,currentsubsection]
	\end{frame}
}

\begin{document}

\begin{frame}[allowframebreaks] \frametitle{Example 1.73}
  \begin{itemize}
  \item Let's apply pumping lemma to prove that
    \begin{equation*}
    B=\{0^n 1^n\mid
    n \geq 0\}
  \end{equation*}
    is not regular
  \item Assume $B$ is regular. From the lemma there is
    $p$ such that \alert{$\forall s$} in the language with $|s| \geq p$ some
    properties hold
  \item Now consider \alert{a particular $s$} in the language
  \begin{equation*}
    s = 0^p 1^p
  \end{equation*}
  We see that $|s| \geq p$.
  By the lemma, $s$ can be split to 
  \begin{equation*}
s = xy z
\end{equation*}
such that
\begin{equation*}
 xy^i z \in B, \forall i \geq 0, \quad
  |y| > 0, \quad \text{ and } |xy| \leq p
\end{equation*}
\item However, we will show that this is not possible
\begin{enumerate}
\item If
  \begin{equation*}
  y=0 \cdots 0
\end{equation*}
then
\begin{equation*}
xy = 0 \cdots 0 \text{ and }
z = 0 \cdots 0 1 \cdots 1
\end{equation*}
Thus
\begin{equation*}
xyyz: \#0 > \#1
\end{equation*}
Then $xy^2 z \notin B$, a contradiction
\item If
  \begin{equation*}
y = 1 \cdots 1,
\end{equation*}
similarly
\begin{equation*}
  xy^2z \notin B \text{ as } \# 0 < \# 1
\end{equation*}

\item If
  \begin{equation*}
y = 0 \cdots 0 1 \cdots 1
\end{equation*}
then
\begin{equation*}
xyyz \notin B \text{ as it is not in the form of } 0^? 1^?
\end{equation*}
\end{enumerate}
\item Therefore, we fail to find $xyz$ with $|y|> 0$ such that
  \begin{equation*}
     xy^i z \in B, \forall i \geq 0
  \end{equation*}
Thus we get a contradiction
\item We see that the condition
  \begin{equation*}
    |xy| \leq p
  \end{equation*}
is not used, but we already reach the contradiction
\item For subsequent examples we will see that this condition is used
\end{itemize}\end{frame}

\begin{frame}[allowframebreaks] \frametitle{Example 1.39}
  \begin{itemize}
\item $C=\{w \mid \#0 = \#1\}$
\item We follow the previous example to have
  \begin{equation*}
  s=0^p 1^p=xyz
\end{equation*}
\item However, we cannot get the needed contradiction for the case
  of
  \begin{equation*}
    y = 0 \cdots 0 1 \cdots 1
  \end{equation*}
\item Earlier we said
  \begin{equation*}
    xyyz \text{ not in the form of } 0^? 1^?
  \end{equation*}
but now we only require
\begin{equation*}
  \# 0 = \# 1
\end{equation*}
\item It is possible that 
  \begin{equation*}
x=\epsilon, z=\epsilon, y = 0^p 1^p 
\end{equation*}
and then
\begin{equation*}
|y| > 0 \text{ and } xy^i z\in C, \forall i
\end{equation*}

\item The 3rd condition should be applied
  \begin{equation*}
|xy| \leq p \Rightarrow y = 0 \cdots 0 \text{ in } s = 0^p 1^p
\end{equation*}
Then 
\begin{equation*}
xyyz \notin C
\end{equation*}
\item Question: the pumping lemma says
  \begin{equation*}
    \forall s \in A, \cdots
  \end{equation*}
  but why in the examples we analyzed a \alert{particular $s$}?
\item And it seems that the selection of $s$ is important. Why?

\item We will explain our use of the lemma in more detail
\end{itemize}\end{frame}


\end{document}
