\input{header}

\AtBeginSubsection[]
{
	\begin{frame}<beamer>
		\frametitle{Outline}
		\tableofcontents[current,currentsubsection]
	\end{frame}
}

\begin{document}

\begin{frame}[allowframebreaks] \frametitle{Complexity}
  \begin{itemize}
  \item From past discussion, we know
    \begin{center}
    decidable $\rightarrow$ computationally solvable
  \end{center}
\item However, this does not mean it is solvable in practice
\item The running time may be just too long
\end{itemize}\end{frame} \begin{frame}[allowframebreaks] \frametitle{Example}
  \begin{itemize}
\item $A=\{0^k 1^k\mid k \geq 0\}$

\item [] What's the \# steps by a 1-tape TM to process a string? 
\item Remember the procedure
  \begin{enumerate}
  \item check if input is $0^* 1^*$
  \item repeat until no 0 or 1

scan, cross off single 0 and 1
\item if 0 or 1 remains, reject

otherwise, accept

  \end{enumerate}
\item How much time?

\item [] Need to count \# steps
\end{itemize}\end{frame} \begin{frame}[allowframebreaks] \frametitle{Analysis}
  \begin{itemize}
\item Worst-case analysis

\item [] Longest time (i.e., largest \# of steps) for all inputs

\item Average-case analysis
\item Usually it is easier to do worst-case analysis
\item We use a function
  \begin{equation*}
  f: N \rightarrow N
\end{equation*}
to represent the number of steps
\item [] $N$: natural number

\item [] $n$: length of input, $f(n)$: number of steps

\end{itemize}\end{frame} \begin{frame}[allowframebreaks] \frametitle{Big-O}
  \begin{itemize}
  \item
A way to understand the running time of the algorithm when
it is run on large inputs
\item Consider
  \begin{equation*}
  f(n)=6n^3 + 5
\end{equation*}
  We have
  \begin{equation*}
n \rightarrow \infty, 6n^3 + 5 \approx 6n^3
\end{equation*}
\item $O(f(n))=O(n^3)$

\item [] How about 6?

  \begin{center}
    \begin{tabular}{l}
$6n^3$ vs. $n^3$ \\
$6n^3$ vs. $n^4$
    \end{tabular}
\end{center}
Only things involved with $n$ are important
\item Definition:
  \begin{equation*}
  f(n)=O(g(n))
\end{equation*}
if 
  \begin{equation*}
    \exists c, n_0, \forall n \geq n_0,
f(n) \leq c g(n).
  \end{equation*}
\end{itemize}\end{frame} \begin{frame}[allowframebreaks] \frametitle{Example}
    \begin{itemize}
    \item Consider
      \begin{equation*}
      f(n) = 6n^3 + 5
    \end{equation*}
    We have
    \begin{equation*}
6n^3 + 5 \leq 7n^3 \text{ after } n \geq 2
\end{equation*}
That is, we choose 
\begin{equation*}
c=7 \text{ and } n_0 = 2
\end{equation*}
Thus 
\begin{equation*}
f(n) = O(n^3)
\end{equation*}
\item $f(n)=O(n^4)$ as 
  \begin{equation*}
6n^3+5 \leq 7n^4, \text{ after } n \geq 2
\end{equation*}
\item But $f(n) \neq O(n^2)$
  \begin{equation*}
  6n^3+5 \leq c n^2
\end{equation*}
cannot always hold because we can choose large $n$ such that
\begin{equation*}
n^3 > cn^2
\end{equation*}
\item Formally we have the following opposite statement of the definition:
\begin{equation*}
  \forall c, n_0, \exists n \geq n_0, f(n) > c g(n)
\end{equation*}
\end{itemize}\end{frame} \begin{frame}[allowframebreaks] \frametitle{Example 7.4}
  \begin{itemize}
  \item Consider
    \begin{equation*}
    f(n)=3n \log_2 n + 5n \log_2 \log_2 n
  \end{equation*}
\item We prove
\begin{equation*}
f(n) = O(n\log n)
\end{equation*}
  \item Note that we write
  \begin{equation*}
  \log n \text{ instead of } \log_2 n
\end{equation*}
as we will show that the result holds for
any base $b$ for the log function
\item Proof: From
  \begin{equation*}
    n \leq 2^n, \forall n \geq 1,
  \end{equation*}
we have
\begin{equation*}
   \log_2 n \leq n
\end{equation*}
From this,
\begin{equation*}
 \log_2 \log_2 n \leq \log_2 n 
\end{equation*}
Therefore
\begin{equation*}
 f(n) \leq 8n \log_2 n = 8n\log_2 b\log_b n, \forall n \geq 1 
\end{equation*}
by using
  \begin{equation*}
\frac{ \log_2 n }{ \log_2 b}= \log_b n
\end{equation*}
\end{itemize}\end{frame} \begin{frame}[allowframebreaks] \frametitle{Other properties}
  \begin{itemize}
  \item We have
    \begin{equation*}
    O(n^2) + O(n) = O(n^2)
  \end{equation*}
\item Formally,
  \begin{equation*}
    \begin{split}
      & f(n) = O(n^2), g(n) = O(n)\\
      & \Rightarrow f(n) + g(n) = O(n^2)
    \end{split}
  \end{equation*}
\item Proof
  \begin{equation*}
    \begin{split}
      & \exists c_1, n_1, \forall n \geq n_1, f(n) \leq c_1 n^2 \\
      & \exists c_2, n_2, \forall n \geq n_2, g(n) \leq c_2 n
    \end{split}
  \end{equation*}
Then
\begin{gather*}
  f(n) + g(n) \leq c_1 n^2 + c_2 n \leq (c_1+c_2) n^2\\
  \text{ after } n \geq \max (n_1, n_2)
\end{gather*}
Thus we choose
\begin{equation*}
  c = c_1 + c_2 \text{ and } n_0 = \max(n_1, n_2)
\end{equation*}
\item Definition:
  \begin{equation*}
f(n) =  2^{O(n)}
\end{equation*}
if
$\exists c, n_0$ such that
\begin{equation*}
f(n) \leq 2^{cn}, \forall n \geq n_0
\end{equation*}
\item $O(1)$: $\exists c, n_0$ such that 
  \begin{equation*}
f(n)   \leq c 1, \forall n \geq n_0
\end{equation*}
Thus
\begin{equation*}
  f(n) \leq
  \max \{f(1), \ldots, f(n_0-1), c\}, \forall n
\end{equation*}
\item That is,
  \begin{center}
  $f(n)$ always $\leq$ a constant
\end{center}
\end{itemize}\end{frame}


\end{document}
