\input{header}

\AtBeginSection[]
{
	\begin{frame}<beamer>
		\frametitle{Outline}
		\tableofcontents[current,currentsubsection]
	\end{frame}
}

\begin{document}

\begin{frame}[allowframebreaks] \frametitle{Regular $\Rightarrow$  a regular
    expression}
  \begin{itemize} 
  \item That is, if given an NFA, how can we convert it to a
    regular expression?
\item Example:

  \begin{center}
    \begin{tikzpicture}
    \node[state, initial] (1) {1};                    
    \node[state, accepting, right of=1] (2) {2};          

    \draw (1) edge[above] node{$b$} (2);
\draw (1) edge[loop above, left] node{$a$} (1);    
\draw (2) edge[loop above, right] node{$a,b$} (2);    
\end{tikzpicture}
  \end{center}
  
  Quickly we see that this corresponds to
  \begin{equation*}
  a^*b(a\cup b)^*
\end{equation*}
\item Example 1.68

\item [] Consider the following DFA

  \begin{center}
    \begin{tikzpicture}
    \node[state, initial] (1) {1};                    
    \node[state, accepting, right of=1] (2) {2};
    \node[state, accepting, below right of=1] (3) {3};              

    \draw (1) edge[bend left=10, above] node{$a$} (2);
    \draw (2) edge[bend left=10, below] node{$a$} (1);
    \draw (1) edge[bend left=10, right] node{$b$} (3);
    \draw (3) edge[bend left=10, below] node{$b$} (1);    
    
    \draw (2) edge[loop above, right] node{$b$} (2);
    \draw (3) edge[right] node{$a$} (2);        
\end{tikzpicture}
  \end{center}
\item It is not that easy to directly see what the regular
  expression is
\item We need a procedure shown below
\item First, add a start and an accept states
    \begin{center}
      \begin{tikzpicture}
        \node[state, initial] (s) {$s$};
    
    \node[state, above right of=s] (1) {1};                    
    \node[state, right of=1] (2) {2};
    \node[state, below right of=1] (3) {3};              
        \node[state, accepting, below right of=2] (a) {$a$};        

    \draw (1) edge[bend left=10, above] node{$a$} (2);
    \draw (2) edge[bend left=10, below] node{$a$} (1);
    \draw (1) edge[bend left=10, right] node{$b$} (3);
    \draw (3) edge[bend left=10, below] node{$b$} (1);    
    
    \draw (2) edge[loop above, right] node{$b$} (2);
    \draw (3) edge[right] node{$a$} (2);        

    \draw (s) edge[above] node{$\epsilon$} (1);        

    \draw (2) edge[above] node{$\epsilon$} (a);        
    \draw (3) edge[below] node{$\epsilon$} (a);

  \end{tikzpicture}
  \end{center}

\item This generates a
\alert{generalized NFA}
\item Our procedure is
  
  \begin{center}
  DFA $\rightarrow$ GNFA $\rightarrow$
regular expression
\end{center}
\item Remove state 1
\begin{center}
\begin{tikzpicture}
  \node[state, initial] (s) {$s$};
    
  \node[state, above right of=s] (2) {2};
  \node[state, below right of=s] (3) {3};              
  \node[state, accepting, below right of=2, xshift=0.8cm] (a) {$a$};        

  \draw (s) edge[above] node{$a$} (2);
  \draw (s) edge[below] node{$b$} (3);        
  \draw (3) edge[bend right=10, right] node{$ba \cup a$} (2);
  \draw (2) edge[bend right=10, left] node{$ab$} (3);
  \draw (2) edge[loop right, right] node{$aa\cup b$} (2);
  \draw (3) edge[loop right, right] node{$bb$} (3);  
    \draw (2) edge[above] node{$\epsilon$} (a);        
    \draw (3) edge[below] node{$\epsilon$} (a);
  \end{tikzpicture}
  \end{center}

\item Example: the link
  \begin{equation*}
    3 \rightarrow 2
  \end{equation*}
  \begin{center}
    \begin{tikzpicture}
  \node[state] (3) {3};
  \node[state, below right of=3] (1) {1};
  \node[state, right of=3, xshift=1cm] (2) {2};                

  \draw (3) edge[above] node{$a$} (2);
  \draw (3) edge[left] node{$b$} (1);
  \draw (1) edge[right] node{$a$} (2);          
    \end{tikzpicture}
  \end{center}
\item Thus $ba \cup a$
  
\item Remove state 2
  \begin{center}
\begin{tikzpicture}
  \node[state, initial] (s) {$s$};
  \node[state, below right of=s] (3) {3};              
  \node[state, accepting, right of=s, xshift=0.8cm] (a) {$a$};        

  \draw (s) edge[above] node{$a(aa\cup b)^*$} (a);          
  \draw (s) edge[left] node{$a(aa\cup b)^* ab \cup b$} (3);        
  \draw (3) edge[loop below, below] node{$(ba \cup a)(aa\cup b)^*ab \cup bb$} (3);  
    \draw (3) edge[right] node{$(ba \cup a)(aa\cup b)^* \cup \epsilon$} (a);
  \end{tikzpicture}
\end{center}

\item Example:
  \begin{equation*}
    s \rightarrow 3
  \end{equation*}

  \begin{center}
    \begin{tikzpicture}
  \node[state] (s) {$s$};
  \node[state, below right of=s] (2) {2};
  \node[state, right of=s, xshift=1cm] (3) {3};                

  \draw (s) edge[above] node{$b$} (3);
  \draw (s) edge[left] node{$a$} (2);
  \draw (2) edge[right] node{$ab$} (3);
  \draw (2) edge[loop below] node{$aa \cup b$} (2);            
    \end{tikzpicture}
  \end{center}
\item Thus $a (aa \cup b)^*ab  \cup b$
\item Here we need to handle
  \begin{equation*}
    2 \xrightarrow{aa \cup b} 2
  \end{equation*}

\item Remove state 3

  \begin{center}
\begin{tikzpicture}
  \node[state, initial] (s) {$s$};
  \node[state, accepting, right of=s, xshift=3cm] (a) {$a$};        

  \draw (s) edge[below] node{} (a);          
  \end{tikzpicture}
\end{center}
\begin{equation*}
  \begin{split}
& (a(aa\cup b)^* ab \cup b)
((ba \cup a)(aa \cup b)^* ab \cup bb)^*
\\
&((ba \cup a)
    (aa\cup b)^* \cup \epsilon) \cup a(aa \cup b)^*
  \end{split}
\end{equation*}
\item We will formally explain the procedure
\end{itemize}
\end{frame}

\begin{frame}[allowframebreaks] \frametitle{Definition of GNFA}
  \begin{itemize}
\item Between any two states: a regular expression
\item $(Q, \Sigma, \delta, q_{start}, \{q_{accept}\})$
  \begin{equation*}
    \delta:(Q-\{q_{accept}\})\times
(Q-\{q_{start}\})\rightarrow R
  \end{equation*}
$R$: all regular expressions over $\Sigma$

% \item States are connected by an R

% \item GNFA accepts $w=w_1 \cdots w_k, w_i \in \Sigma^*$

% \item [] If a sequence of states $q_0 \cdots q_k$ such that
% \begin{enumerate}
% \item $q_0 =q_{start}$
% \item $q_k=q_{accept}$
% \item $R_i=\delta(q_{i-1},q_i), w_i \in L(R_i)$
% \end{enumerate}
\item DFA $\rightarrow $ GNFA

\item [] Two new states: $q_{start}, q_{accept}$

  \begin{center}
    \begin{tabular}{l}
$q_{start} \rightarrow q_0$ with $\epsilon$ \\
any $q \in F \rightarrow q_{accept}$ with $\epsilon$
    \end{tabular}
\end{center}
\item In the definition, between any two states there is
  an expression

\item [] But what if in the graph two states are not connected ?

\item [] $\emptyset \in R$ so if no connection, we simply consider
  $\emptyset$ as the expression between two states
\end{itemize}\end{frame} \begin{frame}[allowframebreaks] \frametitle{GNFA $\rightarrow$ regular expression}
    \begin{itemize}
\item Fig 1.63

  \begin{center}
\begin{tikzpicture}
  \node[state] (qi) {$q_i$};
  \node[state, right of=qi] (qj) {$q_j$};
  \node[state, below right of=qi] (qr) {$q_{\text{rip}}$};  

  \draw (qi) edge[bend right, left] node{$R_1$} (qr);
  \draw (qi) edge[bend left, above] node{$R_4$} (qj);  
  \draw (qr) edge[bend right, right] node{$R_3$} (qj);
  \draw (qr) edge[loop below, right] node{$R_2$} (qr);            
  \end{tikzpicture}
\end{center}


  \begin{center}
\begin{tikzpicture}
  \node[state] (qi) {$q_i$};
  \node[state, right of=qi, xshift=3cm] (qj) {$q_j$};        

  \draw (qi) edge[above] node{$(R_1)(R_2)^*(R_3)\cup (R_4)$} (qj);          
  \end{tikzpicture}
\end{center}
\item $q_{\text{rip}}$ is the state being removed
\item In the procedure 
  
3-state DFA $\rightarrow $ 5-state GNFA
$\rightarrow$ 4-state $\cdots \rightarrow$ 2-state GNFA $\rightarrow$
regular expression
\item In the procedure any any $(i,j)$ related to $q_{\text{rip}}$ considered
\item Algorithm: convert(G)
  \begin{enumerate}
  \item $k$: \# of G
  \item If $k=2$
    \begin{center}
      return $R$
    \end{center}

  \item If $k > 2$, choose any $q_{\text{rip}} \in Q \backslash \{q_s,
    q_a\}$ for removal
\begin{equation*}
  Q'
=Q - \{q_{\text{rip}}\}
\end{equation*}
\begin{equation*}
  \forall q_i \in Q' - \{q_{accept}\},
q_j \in Q' - \{q_{start}\}
\end{equation*}
\begin{equation*}
  \delta'(q_i,q_j)
= R_1 R_2^* R_3 \cup R_4,
\end{equation*}
where
\begin{gather*}
R_1=\delta(q_i,q_{\text{rip}}),
R_2=\delta(q_{\text{rip}},q_{\text{rip}}),
\\
R_3 =\delta(q_{\text{rip}},q_j), 
R_4=\delta(q_i,q_j)
\end{gather*}

\item Run convert$(G')$, where
  \begin{equation*}
    G' =
(Q', \Sigma, \delta', q_s, q_a)
  \end{equation*}
  \end{enumerate}

\item Why in the textbook we modify DFA to GNFA?
\item [] Is it ok to use NFA?

\item [] Seems ok?? 
\end{itemize}\end{frame}

% \begin{frame}[allowframebreaks]

% \frametitle{Formal proof: any GNFA  $G$,
% convert$(G) \equiv G$}
%   \begin{itemize}
% \item 
% by induction on $k$
% \item $k=2$: yes
% \item $k-1 \Rightarrow k$

% $G'$: remove $Q_{\text{rip}}$ from $G$

% Strategy
% \begin{enumerate}
% \item convert$(G')$ = convert$(G)$

% as $G \rightarrow G' \cdots \rightarrow R$

% i.e., they lead to the same regular expression
% \item $G = G'$

% Details: 
% \begin{gather*}
%   w \in G \Rightarrow w \in G'\\
% w \in G' \Rightarrow w \in G
% \end{gather*}
% \item By induction

% $G' = \mbox{convert}(G')$
% \end{enumerate}
% Thus

% $G \equiv G' \equiv
% \mbox{convert}(G')=
% \mbox{convert}(G)$

% \item $G \equiv G'$
% \item $G$ accepts $w
% \Rightarrow G'$ as well

% $G$: $q_{start}q_1 \cdots q_{accept}$

% if no $q_{rid}$ in them

% $\Rightarrow$ $ G'$ accepts $w$

% Though in $G'$, $q_i \rightarrow q_j$
% changed

% Due to union, original still there

% if $q_{rid} \Rightarrow 
% q_i q_{rid} q_j \Rightarrow$ OK
% \item $G'$ accepts $w$
% $\Rightarrow G$ as well

% $q_i \rightarrow q_j$ in $G'$ 
% $\Rightarrow$
% $q_i \rightarrow q_j \mbox{ or }
% q_i \rightarrow q_{\text{rip}} 
% \rightarrow q_j$ in $G$

% \end{itemize}\end{frame}

\end{document}
