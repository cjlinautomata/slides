\input{header}
\AtBeginSection[]
{
	\begin{frame}<beamer>
		\frametitle{Outline}
		\tableofcontents[current,currentsubsection]
	\end{frame}
}

\begin{document}

\begin{frame}[allowframebreaks] \frametitle{Example 1.11}
  \begin{itemize}
  \item Fig 1.12
    \begin{center}
    \begin{tikzpicture}
  \node[state,initial]   (s)                      {$s$};
  \node[state,accepting] (q_1) [below left of=s]  {$q_1$};
  \node[state]           (q_2) [below of=q_1]     {$q_2$};
  \node[state,accepting] (r_1) [below right of=s] {$r_1$};
  \node[state]           (r_2) [below of=r_1]     {$r_2$};

  \path[->]
  (s)   edge [above]             node {a} (q_1)
        edge [above]             node {b} (r_1)
  (q_1) edge [loop left]  node {a} (   )
        edge [bend left,right]  node {b} (q_2)
  (q_2) edge [loop left]  node {b} (   )
        edge [bend left,left]  node {a} (q_1)
  (r_1) edge [loop right] node {b} (   )
        edge [bend left, right]  node {a} (r_2)
  (r_2) edge [loop right] node {a} (   )
        edge [bend left, left]  node {b} (r_1);      
    \end{tikzpicture}
  \end{center}
\item $L(M)=?$
  \begin{equation*}
    a\ldots a, b \ldots b
  \end{equation*}

  where ``$\ldots$'' can be any string of $a$ and $b$
\item First we check that any string accepted by the machine
  must be
    \begin{equation*}
    a\ldots a, b \ldots b
  \end{equation*}
\item Second we check that any
      \begin{equation*}
    a\ldots a, b \ldots b
  \end{equation*}
can be recognized by the machine
\item This machine handles strings with the same character in
  the beginning and in the end
\end{itemize}\end{frame}


\begin{frame}[allowframebreaks] \frametitle{Example 1.13}
  \begin{itemize}
  \item Figure 1.14

    \begin{tikzpicture}[node distance=4cm]
 \node[state,accepting,initial] (q_0) {$q_0$};
  \node[state] (q_1) [above right of=q_0] {$q_1$};
  \node[state] (q_2) [below right of=q_1] {$q_2$};

  \path 
        (q_1) edge[bend right=15, above]  node [rotate = 45] {$2, \langle  \text{reset}\rangle$} (q_0)
  (q_0) edge[bend right=15, above] node {$1$} (q_1)
        (q_0) edge[loop below]  node {$0, \langle  \text{reset}\rangle$} (q_0)
        (q_1) edge[loop above]  node {$0$} (q_1)
        (q_1) edge[bend left=15, above] node {$1$} (q_2)
        (q_2) edge[bend left=15, above] node {$2$} (q_1)
        (q_2) edge[loop right]   node {$0$} (q_2)
        (q_0) edge[bend left=15, above] node {$2$} (q_2)
        (q_2) edge[bend left=15, below] node {$1, \langle  \text{reset}\rangle$} (q_0);    
      \end{tikzpicture}
\item $\Sigma=
\{\langle  reset\rangle,0,1,2\}$
\begin{gather*}
  L(M)=\ldots \ldots\langle  reset\rangle\ldots\langle  reset\rangle\ldots
\\
 =\{\mbox{sum of the
    last segment } \mod 3 = 0\}
\end{gather*}
\item Example:
  \begin{equation*}
    10\langle  reset\rangle22\langle  reset\rangle012
  \end{equation*}
\item An example of running a string
  \begin{equation*}
    \begin{split}
      &      q_0 \xrightarrow{1} q_1 \xrightarrow{0} q_1
      \xrightarrow{\langle reset \rangle} q_0 
            \xrightarrow{2} q_2 \xrightarrow{2} q_1 \\
&      \xrightarrow{\langle reset \rangle} q_0
      \xrightarrow{0} q_0 \xrightarrow{1} q_1 \xrightarrow{2} q_0
    \end{split}
  \end{equation*}
Accepted
\item Each node stores the sum of the current segment
  $\mod 3$
\end{itemize}\end{frame} 

\begin{frame}[allowframebreaks] \frametitle{Formal Definition of Computation}
  \begin{itemize}
\item $M$ accepts $w=w_1 \cdots w_n$ if $\exists$ states

 $r_0 \cdots r_n$ such that
  \begin{enumerate}
  \item $r_0 = q_0$
  \item $\delta(r_i, w_{i+1})= r_{i+1}, i = 0, \ldots, n-1$
  \item $r_n \in F$
  \end{enumerate}
\item \alert{Definition}:
  a language is regular if recognized by some automata
\item This is a very important definition
\item Examples described earlier are regular languages
\item We say \alert{some} automata, so it's possible to have
  several automata for the same language
\item As long as there is one, then the language is regular
\end{itemize}\end{frame} \begin{frame}[allowframebreaks] \frametitle{Designing Automata}
  \begin{itemize}
\item Given a language, how do we construct a machine to recognize it?
\item Basically we need to get a state diagram (where the number
  of states is finite)
\item Earlier we had the opposite: a machine is given and we check the
  corresponding language
\item Example: an automaton recognizing $\{0,1\}$ strings with an odd
\# of 1's 

\item []Fig 1.20

\begin{center}
    \begin{tikzpicture}
\node[state,initial] (q_e) {$q_e$};
\node[state,accepting] (q_o) [right of=q_e] {$q_o$};      

  \path (q_e) edge[bend left, above]  node {$1$} (q_o)
        (q_e) edge[loop above] node {$0$} (q_e)
        (q_o) edge[loop above] node {$0$} (q_o)
        (q_o) edge[bend left, below]  node {$1$} (q_e);
      \end{tikzpicture}
    \end{center}
\item Sample strings

\item [] 01
  \begin{equation*}
    q_e \xrightarrow{0} q_e \xrightarrow{1} q_o
  \end{equation*}
  010101
  \begin{equation*}
    q_e \xrightarrow{0} q_e \xrightarrow{1} q_o
    \xrightarrow{0} q_o \xrightarrow{1} q_e \xrightarrow{0} q_e
    \xrightarrow{1} q_o
  \end{equation*}
  
\item Two ways to think about the design
  \begin{itemize}
  \item After the first 1, we go to $q_o$. Subsequently, every $1, \ldots, 1$ pair is cancelled out by
  \begin{equation*}
q_o \xrightarrow{1} q_e \rightarrow \cdots \rightarrow q_e
    \xrightarrow{1} q_o
  \end{equation*}
\item $q_e, q_o$ respectively remember whether the number of 1's
  so far is even or odd
  \end{itemize}
\item Example 1.21
  
\item [] Language: strings containing 001

\item [] Fig 1.22

\begin{center}
    \begin{tikzpicture}
\node[state,initial] (q) {$q$};
\node[state] (q_0) [right of=q] {$q_0$};
\node[state] (q_00) [right of=q_0] {$q_{00}$};
\node[state,accepting] (q_001) [right of=q_00] {$q_{001}$};      
  \path 
        (q) edge[loop above] node {$1$} (q)
  (q) edge[bend left, above]  node {$0$} (q_0)
  (q_0) edge[bend left, below]  node {$1$} (q)
  (q_0) edge[above]  node {$0$} (q_00)
  (q_00) edge[loop above] node {$0$} (q_00)
  (q_00) edge[above]  node {$1$} (q_001)  
   (q_001) edge[loop below] node {$0,1$} (q_001);
      \end{tikzpicture}
    \end{center}
  \item $q_0, q_{00}$ indicate that before the current input character,
    we have 0 and 00, respectively
  \end{itemize}\end{frame}

\end{document}
