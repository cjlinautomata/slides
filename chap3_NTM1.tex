\input{header}

\AtBeginSubsection[]
{
	\begin{frame}<beamer>
		\frametitle{Outline}
		\tableofcontents[current,currentsubsection]
	\end{frame}
}

\begin{document}

\begin{frame}[allowframebreaks] \frametitle{Nondeterministic TM}
  \begin{itemize}
\item $\delta$:
  \begin{equation*}
    \delta: Q\times \Gamma
\rightarrow P(Q\times \Gamma \times \{L,R\})
  \end{equation*}
$P$: power set
\item Note that following the textbook, we allow only
  $\{L, R\}$ instead of $\{L, S, R\}$
\item Example:
  \begin{eqnarray*}
    q_0,a & \rightarrow & q_1,b,R\\
& \rightarrow & q_2, c, L
  \end{eqnarray*}
\item What if 
  \begin{eqnarray*}
    q_0,a & \rightarrow & q_{accept} \\
& \rightarrow & q_{reject}
  \end{eqnarray*}

\item
  For NTM, by definition $w$ is accepted if one branch works

\item
  In this sense, unless all branches are finite
  \begin{center}
  NTM $\rightarrow$ accept or endless loop
\end{center}
\item
  Thus NTM is like an ``acceptor''

\end{itemize}\end{frame} \begin{frame}[allowframebreaks] \frametitle{Example of NTM}
  \begin{itemize}
\item 
$A = \{w\mid w\mbox{ contains } aab\}$
\item State diagram

  \begin{center}
    \begin{tikzpicture}
  \node[state,initial] (q_0) {$q_0$};
  \node[state,right of=q_0] (q_1) {$q_1$};
  \node[state,right of=q_1] (q_2) {$q_2$};
  \node[state,right of=q_2] (q_a) {$q_a$};
  \node[state,below right of=q_0,yshift=-0.5cm] (q_r) {$q_r$};
  \path (q_0) edge[loop above] node{
      $a, b \rightarrow a, b,$ R
    } (q_0);
  \path (q_0) edge[above] node{
      $a \rightarrow$ R
  } (q_1);
  \path (q_1) edge[above] node{
      $a \rightarrow$ R
  } (q_2);
  \path (q_2) edge[above] node{
      $b \rightarrow$ R
    } (q_a);
  \path (q_0) edge[left] node{
      $\sqcup \rightarrow$ R      
    } (q_r);
  \path (q_1) edge[right] node{
    \begin{tabular}{l}
      $b \rightarrow$ R\\
      $\sqcup \rightarrow$ R      
    \end{tabular}
    } (q_r);
  \path (q_2) edge[right, bend left] node{
    \begin{tabular}{l}
      $a \rightarrow$ R\\
      $\sqcup \rightarrow$ R      
    \end{tabular}
    } (q_r);
    \end{tikzpicture}
  \end{center}
%   \begin{center}
%     \begin{pspicture}(-.2,-.5)(9,4)% \showgrid
%       \psset{radius=.3}
% \rput(.5,2.5){\circlenode{0}{$q_0$}}
% \rput(2.5,2.5){\circlenode{1}{$q_1$}}
% \rput(4.5,2.5){\circlenode{2}{$q_2$}}
% \rput(2.5,0.0){\circlenode{3}{$q_r$}}
% \rput(6.5,2.5){\circlenode{4}{$q_a$}}
% \ncline{->}{0}{1}
% \ncline{->}{1}{2}
% \ncline{->}{2}{4}
% \ncline{->}{0}{3}
% \ncline{->}{1}{3}
% \ncline{->}{2}{3}
% \pnode(-.2,2.5){B}
% \ncline{->}{B}{0}
% \rput(1.4,2.7){$a \rightarrow R$}
% \rput(3.4,2.7){$a \rightarrow R$}
% \rput(5.4,2.7){$b \rightarrow R$}
% \rput(0.6,1.5){$\sqcup \rightarrow R$}
% \rput(3.1,1.5){$\sqcup \rightarrow R$}
% \rput(3.1,1.8){$b \rightarrow R$}
% \rput(4.5,1.5){$\sqcup \rightarrow R$}
% \rput(4.5,1.8){$a \rightarrow R$}
% \nccircle{<-}{0}{.4}
% \rput(0.5,3.6){$a,b\rightarrow a,b,R$}
%     \end{pspicture}
%   \end{center}

  
\item You may recall that this is an NFA example discussed before
  
\item Only the first node is nondeterministic

\end{itemize}\end{frame} \begin{frame}[allowframebreaks] \frametitle{Example of NTM}
  \begin{itemize}
\item $L=\{0^n\mid n \mbox{ composite number}\}$
\item From p. 204 of Lewis and Papadimitriou
\item Composite number: product of two natural numbers
\item Procedure
  \begin{itemize}
  \item Nondeterministically choose $p$ and $q$

  \item [] Sequentially try $p$ from $2$ to $n-1$

  \item Check if $n=pq$

  \item [] This can be done by the earlier example
    \begin{equation*}
    \{a^n b^p c^q\mid n = p \times q\}
  \end{equation*}
  \end{itemize}
\item Question: details about ``non-deterministically'' choose $p$ and $q$?
\item If we sequentially try all $(p,q)$ combinations, then
  looks like we have a deterministic setting?
  
\item Our generation of  $p$ and $q$ can be non-deterministic
\item Say we do a copy operation to generate $p$ elements. The TM can
  stop this operation at any time point
\end{itemize}\end{frame}

\end{document}

%%% Local Variables:
%%% mode: latex
%%% TeX-master: t
%%% End:

