\input{header}

\AtBeginSubsection[]
{
	\begin{frame}<beamer>
		\frametitle{Outline}
		\tableofcontents[current,currentsubsection]
	\end{frame}
}

\begin{document}

\begin{frame}[allowframebreaks] \frametitle{Proving Languages
    not Decidable}
\begin{itemize}
\item Earlier we proved that $A_{\text{TM}}$ (i.e., the acceptance problem of
    TM) is undecidable
  \item Many other problems are not decidable either
  \item But how to prove that?
  \item We will introduce a technique
    called \alert{reducibility}
  \end{itemize}\end{frame}

\begin{frame}[allowframebreaks] \frametitle{Reducibility}
  \begin{itemize}
  \item Reduction: converting the first problem to the
    second so the second can be used to solve the first
  \item Example: finding your way around a city can be solved
    by having a map
  \item [] The problem is reduced to obtaining a map
  \item We say $A$ reduces to $B$ if we can use a solution to
    $B$ to solve $A$
  \item In the earlier example,
    \begin{center}
      \begin{tabular}{l}
        $A$: getting around a city\\
        $B$: obtaining a map
      \end{tabular}
    \end{center}      
    \item A mathematical example:
      \begin{center}
        \begin{tabular}{l}
          $A$: measuring the area of a rectangle\\
          $B$: measuring the length/width
        \end{tabular}
      \end{center}
    \item When $A$ is reducible to $B$, solving $A$ is
      not harder than $B$
    \item The reason is that a solution to $B$ gives
      a solution to $A$
    \item Therefore,
      \begin{center}
        $B$ decidable $\Rightarrow$ $A$ decidable
      \end{center}
      and
      \begin{center}
        $A$ undecidable $\Rightarrow$ $B$ undecidable
      \end{center}
\end{itemize}\end{frame} 

\begin{frame}[allowframebreaks]
\frametitle{$E_{\text{TM}}$ Undecidable}
\begin{itemize}
\item Consider  
  \begin{equation*}
    E_{\text{TM}} = \{
\langle  M\rangle \mid M: \mbox{ a TM and } L(M) = \emptyset\}    
  \end{equation*}
\item We prove that it is undecidable
\item Idea: we do the proof by contradiction
\item Assume $E_{\text{TM}}$ is decidable. Then there is a decider
  $R$
\item From $R$ we construct a decider $S$ for $A_{\text{TM}}$. This causes
   a contradiction because $A_{\text{TM}}$ is undecidable
 \item For $A_{\text{TM}}$, the input is $\langle M, w \rangle$. Under such an input, we design $S$ to be as follows
   \begin{enumerate}
   \item Design a TM $M_1$ such that
     \begin{equation}
       \label{eq:M1}
       L(M_1) \neq \emptyset \Leftrightarrow M \text{ accepts } w
     \end{equation}
   \item Run $R$ on input $\langle M_1\rangle$
   \item If $R$ accepts, {\em reject}; if $R$ rejects, {\em accept}
   \end{enumerate}
 \item We discuss the design of $M_1$ later
 \item We see that if $R$ accepts, then $L(M_1) = \emptyset$
   implies that $M$ rejects $w$
 \item Thus for input $\langle M, w \rangle$, in a finite
   number of steps we know if $M$ accepts $w$ or not
 \item Then $A_{\text{TM}}$ is decidable, a contradiction
 \item How to design $M_1$?
 \item $M_1$ takes input $x$ and have
   \begin{enumerate}
   \item If $x \neq w$, {\em reject}
   \item If $x = w$, run $M$ on input $w$ and {\em accept} if $M$ does
   \end{enumerate}
 \item Clearly,
   \begin{equation*}
     L(M_1) = \emptyset \text{ or } \{w\}
   \end{equation*}
   We see that
   \begin{equation*}
     \begin{split}
     &  M \text{ accepts } w \Rightarrow L(M_1) = \{w\} \neq \emptyset\\
     &  L(M_1) \neq \emptyset \Rightarrow M \text{ accepts } w
     \end{split}
   \end{equation*}
 Thus the condition \eqref{eq:M1} is satisfied
 \item $M_1$ takes $w$ as part of its description
 \item This is fine as we can design a machine related to a
   specific string
\end{itemize}
\end{frame}
\end{document}

%%% Local Variables:
%%% mode: latex
%%% TeX-master: t
%%% End:
