\input{header}

\AtBeginSubsection[]
{
	\begin{frame}<beamer>
		\frametitle{Outline}
		\tableofcontents[current,currentsubsection]
	\end{frame}
}

\begin{document}

\begin{frame}[allowframebreaks] \frametitle{Equivalence with context-free grammars}
  \begin{itemize}  
\item  Language context free
$\Leftrightarrow$ recognized by pushdown automata

\item For the left-hand side, recall that by definition a language is context-free if
  it is constructed by some CFG
\item Thus, the proof is equivalent to 
\item []
  Language by CFG
$\Leftrightarrow$ recognized by pushdown automata
\item For the proof, one direction is easier, while the other is harder
\item As usual, we do the easier one ($\Rightarrow$) first
  
\end{itemize}\end{frame} \begin{frame}[allowframebreaks] \frametitle{CFL $\rightarrow$ PDA}
    \begin{itemize}
    \item Given a CFG, we find a PDA to simulate this grammar
      
\item Two keys:

\item [] stack 

\item [] nondeterminism: different substitutions

  
\item We do the proof by an example
\item Suppose we are given the following
  CFG
  \begin{equation*}
    \begin{split}
      & S \rightarrow aTb \mid b\\
      & T \rightarrow Ta \mid \epsilon
    \end{split}
  \end{equation*}

\item Idea: for rule substitution, we replace the left-hand side variable with
  the right-hand side string
\item That is, in PDA, we
  \begin{center}
    pop up the left-hand side variable
  \end{center}
  and
  \begin{center}
push right-hand side to stack
\end{center}
in \alert{a reversed way}
\item For example, we have
  \begin{equation*}
  S \rightarrow aTb    
  \end{equation*}
\item [] Then
  \begin{center}
    $S$ is popped
  \end{center}
and
\begin{center}
  $b, T, a$
\end{center}
are sequentially pushed

  \item
  A PDA can be as follows

\begin{tikzpicture}
\node[state,initial] (q_1) {$q_s$};
\node[state] (q_2) [below of=q_1,yshift=1.2cm] {};
\node[state] (q_3) [below of=q_2,yshift=1.2cm] {$q_{\text{loop}}$};
\node[state] (q_4) [above right of=q_3, yshift=-0.5cm, xshift=0.5cm] {};
\node[state] (q_6) [right of=q_4, yshift=1.8cm] {};
\node[state] (q_5) [below right of=q_3, yshift=0.5cm, xshift=2.5cm] {};
\node[state] (q_7) [below of=q_3,yshift=1.2cm,accepting] {$q_a$};
  \path 
  (q_1) edge[left]  node {$\epsilon, \epsilon \rightarrow \$ $} (q_2)
  (q_2) edge[left]  node {$\epsilon, \epsilon \rightarrow S $} (q_3)
  (q_3) edge[left]  node {$\epsilon, \$ \rightarrow \epsilon $} (q_7)    
  (q_3) edge[loop left]  node [xshift=-0.5cm] {
    \begin{tabular}{l}
      $\epsilon,S \rightarrow b$\\
      $\epsilon, T \rightarrow \epsilon$\\
      $a, a \rightarrow \epsilon$\\
      $b, b \rightarrow \epsilon$
    \end{tabular}
  } (q_3)
  (q_3) edge[right]  node {$\epsilon, S \rightarrow b$} (q_4)
  (q_4) edge[right]  node {$\epsilon, \epsilon \rightarrow T$} (q_6)
  (q_6) edge[bend right, above]  node {$\epsilon, \epsilon \rightarrow a$} (q_3)
  (q_3) edge[below, bend left]  node {$\epsilon, T \rightarrow a$} (q_5)
  (q_5) edge[below, bend left]  node {$\epsilon, \epsilon \rightarrow T$} (q_3)  
;
  \end{tikzpicture}

% \item Fig 2.22

% inermediate string: 01A1A0

% substitute A by ??

% check if finally it matches \$
\item We use $\$$ to ensure that before accepting
  any string, stack is empty
\item Then the start variable $S$ is pushed
\item The state $q_{\text{loop}}$ is the main place to handle rules and process input characters
\item Besides rules from CFG, we need
  \begin{equation*}
    \begin{split}
      & a, a \rightarrow \epsilon\\
      & b, b \rightarrow \epsilon.
    \end{split}
  \end{equation*}
Otherwise, input characters are never processed
\item Consider an example sequence $aaaab$
% Draw a tree process $aaaa$
% Before the first $a$, we will see $\infty$
% possibilities. Can handle this situation
% by confine the stack length to be smaller
% than the input length
\end{itemize}

\begin{eqnarray*}
&& q_{\text{start}} \stackrel{\epsilon}{\rightarrow}
q_{\text{loop}}, \{S,\$\} 
\stackrel{\epsilon}{\rightarrow} q_1, \{b,\$\}
\stackrel{\epsilon}{\rightarrow} q_2, \{T,b,\$\} \\
&& \stackrel{\epsilon}{\rightarrow} q_{\text{loop}}, \{a,T,b,\$\} 
\stackrel{a}{\rightarrow}
 q_{\text{loop}}, \{T,b,\$\} \\
&& \stackrel{\epsilon}{\rightarrow} q_{3}, \{a, b,\$\} 
\stackrel{\epsilon}{\rightarrow} q_{\text{loop}}, \{T,a,b,\$\}\\
&& \stackrel{\epsilon}{\rightarrow} q_3, \{a,a,b,\$\}
\stackrel{\epsilon}{\rightarrow} q_{\text{loop}}, \{T,a,a,b,\$\}\\
&& \stackrel{\epsilon}{\rightarrow} q_3, \{a,a,a,b,\$\}
\stackrel{\epsilon}{\rightarrow} q_{\text{loop}}, \{T,a,a,a,b,\$\}\\
&& \stackrel{\epsilon}{\rightarrow} q_{\text{loop}}, \{a,a,a,b,\$\}
\stackrel{a}{\rightarrow} q_{\text{loop}}, \{a,a,b,\$\}\\
&& \stackrel{a}{\rightarrow} q_{\text{loop}},\{a,b,\$\}
\stackrel{a}{\rightarrow} q_{\text{loop}}, \{b,\$\}\\
&& \stackrel{b}{\rightarrow} q_{\text{loop}}, \{\$\}
\stackrel{\epsilon}{\rightarrow} q_{accept}
\end{eqnarray*}
\begin{itemize}
\item 
  Even with a non-deterministic setting, we ensure that only strings generated
  by this CFG can be accepted by the PDA
  \begin{enumerate}
  \item A string is accepted only if all characters are processed
    (this is part of the PDA definition!)
  \item We have $\$$ to ensure that the stack is empty in the end
  \end{enumerate}
\end{itemize}
\end{frame}

\end{document}
