\input{header}

\AtBeginSubsection[]
{
	\begin{frame}<beamer>
		\frametitle{Outline}
		\tableofcontents[current,currentsubsection]
	\end{frame}
}

\begin{document}

\begin{frame}[allowframebreaks] \frametitle{Small-o}
  \begin{itemize}
  \item Two different concepts:

  \item [] $O$: no more than something

  \item [] $o$: less than something
    
\item Definition 
  \begin{equation*}
  f(n)=o(g(n))
\end{equation*}
  if
  \begin{equation*}
    \lim_{n\rightarrow \infty} \frac{f(n)}{g(n)} = 0.
  \end{equation*}
\item The definition of this limit:
\begin{equation*}
  \forall c > 0, \exists n_0, \forall n \geq n_0,
\frac{f(n)}{g(n)} \leq c.
\end{equation*}
\item Note that we may instead write
  \begin{equation*}
\frac{f(n)}{g(n)} < c
\end{equation*}
but these two limit definitions are equivalent
\item $O$ versus $o$:
\begin{equation*}
  \begin{split}
&  \exists c > 0, \exists n_0, \forall n \geq n_0,
f(n) \leq c g(n)\\    
&  \forall c > 0, \exists n_0, \forall n \geq n_0,
f(n) \leq c g(n)
\end{split}
\end{equation*}
The $\forall c$ causes $o$ to be something smaller
\item $\sqrt{n}= o(n)$
  \begin{equation*}
    \lim_{n \rightarrow \infty} \frac{\sqrt{n}}{n}
=
   \lim_{n \rightarrow \infty} \frac{1}{\sqrt{n}}=0
  \end{equation*}
\item $f(n) 
\neq o(f(n))$
\begin{equation*}
    \lim_{n \rightarrow \infty} \frac{f(n)}{f(n)} = 1 \neq 0
\end{equation*}

\end{itemize}\end{frame}

\begin{frame}[allowframebreaks] \frametitle{Example: $A=\{0^k 1^k\mid k \geq 0\}$}
\begin{itemize}
\item Let's count the number of steps in the algorithm
  discussed before
  
\item Check if the input is
  \begin{equation*}
0...0 1...1
\end{equation*}
This takes $O(n)$
\item Move back: $O(n)$
\item Cross off each 0 and 1: $O(n)$

\item [] How many such crosses: $n/2$
  \begin{equation*}
n/2 \times O(n) = O(n^2)
\end{equation*}
\item Accept or not?

\item [] $O(n)$ to go through from beginning to end
\item Total:
  \begin{equation*}
O(n) + O(n^2) +O(n) = O(n^2)
\end{equation*}
\end{itemize}\end{frame}

\begin{frame}[allowframebreaks] \frametitle{Time complexity class}
  \begin{itemize}
  \item Definition:
    \begin{equation*}
      \begin{split}
& \text{TIME}(t(n))\\
\equiv
& \{ L\mid L \text{ a language decided by an $O(t(n))$ TM}
\}
\end{split}
\end{equation*}
\item We have
  \begin{equation*}
  \{0^k 1^k \mid k \geq 0\} \in \text{TIME}(n^2)
\end{equation*}
Can we make it faster?

\end{itemize}\end{frame} \begin{frame}[allowframebreaks] \frametitle{New Algorithm for $A=\{0^k 1^k\mid k \geq 0\}$}
  \begin{itemize}
\item The procedure: cross off every other 0 and 1
\item []$\underline{0}0
\underline{0}0
\underline{0}
\underline{1}1
\underline{1}1
\underline{1}
$
\item []
$\underline{0}0
\underline{1}1
$
\item []
$\underline{0}
\underline{1}
$
\item []
$\epsilon$

\item [] key: length of the string left must be always even

\item A failed algorithm

\item []
$\underline{0}0
\underline{0}0
\underline{1}1
$
\item []
$001
$

\item 
Algorithm
\begin{enumerate}
\item check 0...0 1...1
\item repeat if not empty 

total \# 0 \& 1: odd $\Rightarrow$ reject

cross off every other 0 and 1
\item no 0 \& 1 remain, accept
\end{enumerate}
\item If  13 ``0'' $\Rightarrow$
6 ``0'' $\Rightarrow$
3 ``0'' $\Rightarrow$
1 ``0'' 

\item [] $1+\log_2 n$ iterations

\item Each iteration: $O(n)$ operations
\item [] Note that length of tape contents is still $n$ as we only ``mark'' elements
\item Total cost: $O(n \log n)$

\item Therefore
  \begin{equation*}
\{0^k 1^k\mid k \geq 0\} \in \text{TIME}(n \log n)
\end{equation*}
\item Can we do better? no

\item Any language decided in $o(n\log n)$
on a single-tape TM $\Rightarrow$ regular
(not proved here)
\item But we know that
  \begin{equation*}
  \{0^k1^k \mid k \geq 0\}
\end{equation*}
is not regular
\item What if we copy the remained string to be after the current string?
  It seems that we then have
  \begin{equation*}
    n + \frac{n}{2} + \frac{n}{4} + \cdots = O(n)??
  \end{equation*}
\item The problem is that \alert{the copy operation is expensive}.
  Copying $n$ elements needs $O(n^2)$
  
\end{itemize}\end{frame} \begin{frame}[allowframebreaks] \frametitle{Using two-tape TM for $ \{0^k1^k \mid k \geq 0\}$}
  \begin{itemize}
\item We can have an $O(n)$ procedure
  \begin{enumerate}
  \item check 0...0 1...1

  \item copy 0 to the second tape

find the first 1
\item sequentially cut 1 and 0

if no ``0'' reject
\item if ``1'' left, reject

otherwise, accept
  \end{enumerate}

\item Each step $O(n)$

\end{itemize}\end{frame} 

\end{document}
